\documentclass{article}
\usepackage[answerdelayed, lastexercise]{exercise} %
\usepackage{calc}
\setcounter{tocdepth}{2}
\providecommand{\keywords}[1]{\textbf{\textit{Keywords---}} #1}
\usepackage{tikz-cd}
\usepackage{tensor}
% \usepackage{hyperref}
% \usepackage{csquotes}
\usepackage{fancyvrb}
\usepackage{todonotes}
\usepackage{verbatim}
\usepackage{amsmath}
\usepackage{microtype}
\usepackage{amssymb}
\usepackage{amsthm}
\usepackage{thmtools}
\usepackage{nameref,hyperref}
\usepackage{cleveref}
\usepackage[utf8]{inputenc}

\setcounter{tocdepth}{5}

\newcommand{\ts}[2]{{_{#1}}\!\!\times_{#2}\!}
\newcommand{\tens}[2]{{_{#1}}\times_{#2}}

\declaretheorem[numberwithin=section]{theorem}
\declaretheorem[sibling=theorem]{exercise}
\declaretheorem[sibling=theorem]{example}
\declaretheorem[sibling=theorem]{non-example}
\declaretheorem[sibling=theorem]{lemma}
\declaretheorem[sibling=theorem]{corollary}
\declaretheorem[sibling=theorem]{proposition}
\declaretheorem[style=definition,sibling=theorem]{definition}
\declaretheorem[style=definition,sibling=theorem]{axiom}
\declaretheorem[style=definition,sibling=theorem]{notation}
\declaretheorem[sibling=theorem]{question}
\declaretheorem[style=remark,sibling=theorem]{remark}
\title{Final examination question 25}
\author{Work in progress}
\begin{document}
\maketitle

\begin{Exercise}
  Consider the dynamical system $v_{k+1} = A v_k$ where
  \begin{equation*}
  A = \left[
  \begin{array}{cc}
  0&1\\
  \frac{1}{3} & \frac{2}{3}
  \end{array}
  \right], \left[
  \begin{array}{c}
  x_k\\
  y_k
  \end{array}
  \right]\text{ and } v_0 = \left[
  \begin{array}{c}
  -1\\
  1
  \end{array}
  \right]
  \end{equation*}
  It turns out that $A$ is diagonalisable, that is $P^{-1}AP = D$ is diagonal, where
  \begin{equation*}
  P= \left[
  \begin{array}{cc}
  1&-3\\
  1&1
  \end{array}
  \right]\text{ and }
  D = \left[
  \begin{array}{cc}
  1&0\\
  0&\frac{-1}{3}
  \end{array}
  \right]
  \end{equation*}
  What is the value of $x_k$?
\end{Exercise}

\begin{Answer}
  We are given that $A = PDP^{-1}$ so 
  \begin{equation*}
  \left[
  \begin{array}{c}
  1\\
  1
  \end{array}
  \right]
  \end{equation*} is an eigenvector with eigenvalue $1$ and 
  \begin{equation*}
  \left[
  \begin{array}{c}
  -3\\
  1
  \end{array}
  \right]
  \end{equation*} is an eigenvector with eigenvalue $\frac{-1}{3}$.
  First write the initial vector as a linear combination of the eigenvectors:
  \begin{equation*}
  \left[
  \begin{array}{c}
  -1\\
  1
  \end{array}
  \right] = a \left[
  \begin{array}{c}
  1\\
  1
  \end{array}
  \right]+b \left[
  \begin{array}{c}
  -3\\
  1
  \end{array}
  \right]
  \end{equation*}
  I.e. solve 
  \begin{equation*}
  \left[
  \begin{array}{cc|c}
  1&-3&-1\\
  1&1&1
  \end{array}
  \right]
  \end{equation*}
  \begin{equation*}
  \rightarrow\left[
  \begin{array}{cc|c}
  1&-3&-1\\
  0&4&2
  \end{array}
  \right]
  \end{equation*}
  \begin{equation*}
  \rightarrow\left[
  \begin{array}{cc|c}
  1&-3&-1\\
  0&1&\frac{1}{2}
  \end{array}
  \right]
  \end{equation*}
  \begin{equation*}
  \rightarrow\left[
  \begin{array}{cc|c}
  1&0& \frac{1}{2}\\
  0&1&\frac{1}{2}
  \end{array}
  \right]
  \end{equation*}
  So $a = \frac{1}{2}$ and $b = \frac{1}{2}$.
  Therefore
  \begin{align*}
    v_k &= A^kv_0\\
    &=A^k\left[ \frac{1}{2} \left[
    \begin{array}{c}
    1\\
    1
    \end{array}
    \right]+\frac{1}{2} \left[
    \begin{array}{c}
    -3\\
    1
    \end{array}
    \right]\right] \\
    & = \frac{1}{2}A^k \left[
    \begin{array}{c}
    1\\
    1
    \end{array}
    \right]+\frac{1}{2} A^k \left[
    \begin{array}{c}
    -3\\
    1
    \end{array}
    \right]\\
    &= \frac{1}{2} \left[
    \begin{array}{c}
    1\\
    1
    \end{array}
    \right]+\frac{1}{2}\left(\frac{-1}{3}\right)^k \left[
    \begin{array}{c}
    -3\\
    1
    \end{array}
    \right] \\
    &= \left[
    \begin{array}{c}
    x_k\\
    y_k
    \end{array}
    \right]
  \end{align*}
   and so 
   \begin{equation*}
   x_k = \frac{1}{2}\left(1-3\left(\frac{-1}{3}\right)^k\right)
   \end{equation*}
\end{Answer}

\shipoutAnswer

\bibliography{references}
\bibliographystyle{hplain}

\end{document}
