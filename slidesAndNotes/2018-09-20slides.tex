\documentclass{beamer}

\useoutertheme[subsection=false]{miniframes}
\usecolortheme{beaver}
\setbeamertemplate{navigation symbols}{}
\setbeamertemplate{footline}{}
\usepackage{graphicx}
\usepackage{url}
\usepackage{datetime}

\newcommand {\framedgraphic}[3] {
  \begin{frame}{#1}
    \vspace{-0.5cm}
    \begin{center}
      \includegraphics[width=0.9\textwidth,keepaspectratio]{#2}
    \end{center}
    \vspace{-1cm}
    \begin{center}
      #3
    \end{center}
  \end{frame}
}
\newcommand{\lectureDate}{\formatdate{20}{09}{2018}}

\setbeamertemplate{caption}{\raggedright\insertcaption\par}
\title{MATH211: Linear Methods I}
\author{Matthew Burke}
\date{\lectureDate}
\begin{document}

\frame{\titlepage}

\begin{frame}{Lecture on \lectureDate}
  \tableofcontents
\end{frame}

\section*{Last time}
\label{sec:Last-time}

\begin{frame}{Last time}
  \begin{itemize}
  \item (i,j)-entry proofs\vfill
  \item Identity matrices\vfill
  \item Inverse matrices\vfill
  \item Small examples of inverses
  \end{itemize}
\end{frame}

\section{Matrix inversion algorithm}

\begin{frame}
  \begin{beamercolorbox}[sep=12pt,center]{part title}
    \usebeamerfont{section title}\insertsection\par
  \end{beamercolorbox}
\end{frame}

\begin{frame}{Standard basis vectors}
  \begin{definition}
    The standard basis vectors in $n$-space are:
    \begin{equation*}
      \left[
	\begin{array}{c}
          1\\
          0\\
          0\\
          \vdots\\
          0
	\end{array}
      \right],
      \left[
	\begin{array}{c}
          0\\
          1\\
          0\\
          \vdots\\
          0
	\end{array}
      \right],
      \left[
	\begin{array}{c}
          0\\
          0\\
          1\\
          \vdots\\
          0
	\end{array}
      \right],\dots,
      \left[
	\begin{array}{c}
          0\\
          0\\
          0\\
          \vdots\\
          1
	\end{array}
      \right]
    \end{equation*}
    I.e. the columns of $I_n$.
  \end{definition}
\end{frame}

\begin{frame}{Aim of matrix inversion}
  Given $B$ find $A$ such that $BA=I_n$:\vfill
  \begin{equation*}
    \left[
      \begin{array}{cccc}
        B \left[
	\begin{array}{c}
          a_{11}\\
          a_{21}\\
          \vdots\\
          a_{m1}
	\end{array}
        \right]
        B \left[
	\begin{array}{c}
          a_{12}\\
          a_{22}\\
          \vdots\\
          a_{m2}
	\end{array}
        \right]
        \dots
        B \left[
	\begin{array}{c}
          a_{1n}\\
          a_{2n}\\
          \vdots\\
          a_{mn}
	\end{array}
        \right]
      \end{array}
    \right]= \left[
      \begin{array}{cccc}
        1 & 0 & \dots & 0\\
        0 & 1 & \dots & 0\\
        \dots & \dots & \dots & \dots\\
        0 & 0 & \dots & 1
      \end{array}
    \right]
  \end{equation*}\vfill
  \bf{I.e. find linear combinations of the columns of $B$ that give each of the standard basis vectors.}
\end{frame}

\begin{frame}{The algorithm}
  For the first standard basis vector we solve:
  \begin{equation*}
    Bx = \left[
      \begin{array}{c}
        1\\
        0\\
        \vdots\\
        0
      \end{array}
    \right]\leftrightarrow
    \left[
      \begin{array}{cccc|c}
        b_{11} & b_{12} & \dots & b_{1m} & 1\\
        b_{21} & b_{22} & \dots & b_{2m} & 0\\
        \dots & \dots & \dots & \dots & 0\\
        b_{p1} & b_{p2} & \dots & b_{pm} & 0
      \end{array}
    \right]
  \end{equation*}
  and we can group all together as:
  \begin{equation*}
    \left[
    \begin{array}{cccc|cccc}
      b_{11} & b_{12} & \dots & b_{1m} & 1 & 0 & \dots & 0\\
      b_{21} & b_{22} & \dots & b_{2m} & 0 & 1 & \dots & 0\\
      \dots & \dots & \dots & \dots & \dots & \dots & \dots & \dots\\
      b_{p1} & b_{p2} & \dots & b_{pm} &0 & 0 & \dots & 1
    \end{array}
    \right]
  \end{equation*}
\end{frame}

\begin{frame}{Reduction}
  Row reduce
  \begin{equation*}
    \left[
    \begin{array}{cccc|cccc}
      b_{11} & b_{12} & \dots & b_{1m} & 1 & 0 & \dots & 0\\
      b_{21} & b_{22} & \dots & b_{2m} & 0 & 1 & \dots & 0\\
      \dots & \dots & \dots & \dots & \dots & \dots & \dots & \dots\\
      b_{p1} & b_{p2} & \dots & b_{pm} &0 & 0 & \dots & 1
    \end{array}
    \right]
  \end{equation*}
  to
  \begin{equation*}
    \left[R|A\right]=
    \left[
      \begin{array}{cccc|cccc}
        1 & 0 & \dots & \star & a_{11} & a_{12} &\dots & a_{1n}\\
        0 & 1 & \dots & \star & a_{21} & a_{22} &\dots & a_{2n}\\
        0 & 0 & \dots & \star & \dots & \dots &\dots & \dots\\
        0 & 0 & \dots & \star & a_{m1} & a_{m2} &\dots & a_{mn}
      \end{array}
    \right]
  \end{equation*}
  where $R$ is in reduced row echelon form.
\end{frame}

\begin{frame}{Finishing off}
  \begin{equation*}
    \left[R|A\right]=
    \left[
      \begin{array}{cccc|cccc}
        1 & 0 & \dots & \star & a_{11} & a_{12} &\dots & a_{1n}\\
        0 & 1 & \dots & \star & a_{21} & a_{22} &\dots & a_{2n}\\
        0 & 0 & \dots & \star & \dots & \dots &\dots & \dots\\
        0 & 0 & \dots & \star & a_{m1} & a_{m2} &\dots & a_{mn}
      \end{array}
    \right]
  \end{equation*}\vfill
  If $R$ is:-\vfill
  \begin{description}
  \item [the identity matrix] then $B$ is invertible with inverse $A$\vfill
  \item [anything else] then $B$ is not invertible
  \end{description}
\end{frame}

\begin{frame}{Relationship to solutions of systems}
  Recall that a system of equations is equivalently a matrix equation:
  \begin{equation*}
    Ax=b
  \end{equation*}
  so if $A$ is invertible
  \begin{equation*}
    x = A^{-1}Ax = A^{-1}b
  \end{equation*}
  is the \emph{unique solution} to $Ax=b$.\vfill
  \bf{This method is rarely the most efficient thing to do.}
\end{frame}

\begin{frame}
  Questions?
\end{frame}

\section{Examples}

\begin{frame}
  \begin{beamercolorbox}[sep=12pt,center]{part title}
    \usebeamerfont{section title}\insertsection\par
  \end{beamercolorbox}
\end{frame}

\begin{frame}{Low dimensional examples}
  \begin{example}
    Find the inverse for
    \begin{equation*}
      \left[
        \begin{array}{cc}
          1&2\\
          3&4
        \end{array}
      \right]
    \end{equation*}
  \end{example}
  \begin{example}
    Find the inverse for the general 2x2 matrix
    \begin{equation*}
      \left[
        \begin{array}{cc}
          a&b\\
          c&d
        \end{array}
      \right]
    \end{equation*}
  \end{example}
\end{frame}

\begin{frame}{Examples}
  \begin{example}
    Find the inverse for
    \begin{equation*}
      \left[
        \begin{array}{cc}
          2&4\\
          1&1
        \end{array}
      \right]
    \end{equation*}
  \end{example}
  \begin{example}
    Find the inverse for
    \begin{equation*}
      \left[
        \begin{array}{cc}
          5&4\\
          10&8
        \end{array}
      \right]
    \end{equation*}
  \end{example}
\end{frame}

\begin{frame}{Examples}
  \begin{example}
    Find the inverse for
    \begin{equation*}
      \left[
        \begin{array}{ccc}
          1&0&3\\
          2&3&4\\
          1&0&2
        \end{array}
      \right]
    \end{equation*}
  \end{example}
  \begin{example}
    Find the inverse for
    \begin{equation*}
      \left[
        \begin{array}{ccc}
          1&0&-1\\
          -2&1&3\\
          -1&1&2
        \end{array}
      \right]
    \end{equation*}
  \end{example}
\end{frame}

\begin{frame}
  Questions?
\end{frame}

\section{Common Inverses}

\begin{frame}
  \begin{beamercolorbox}[sep=12pt,center]{part title}
    \usebeamerfont{section title}\insertsection\par
  \end{beamercolorbox}
\end{frame}

\begin{frame}{Inverse of transposes and products}
  \begin{example}
    Prove that $(BA)^T = A^T B^T$.
  \end{example}\vfill
  \begin{example}
    Using the previous example prove that
    \begin{description}
    \item [if] $A$ has inverse $A^{-1}$
    \item [then] $A^T$ has inverse $(A^{-1})^T$.
    \end{description}
  \end{example}\vfill
  \begin{example}
    Prove that
    \begin{description}
    \item [if] $A$ has inverse $A^{-1}$ and $B$ has inverse $B^{-1}$
    \item [then] $BA$ has inverse $A^{-1}B^{-1}$.
    \end{description}
  \end{example}
\end{frame}

\begin{frame}{Elementary matrices}
  Recall the elementary row operations:\vfill
  \begin{enumerate}
  \item Swap two rows.
  \item Multiply a row by a non-zero scalar.
  \item Add a multiple of one row to another.
  \end{enumerate}\vfill
  Each corresponds to multiplying on the left\vfill
  \begin{equation*}
    A \mapsto EA
  \end{equation*}\vfill
  by a different \emph{elementary matrix $E$}.
\end{frame}

\begin{frame}{Multiply a row by a non-zero scalar}
  The matrix that corresponds to multiplying the second row by $3$ is:\vfill
  \begin{equation*}
    \left[
      \begin{array}{ccc}
        1&0&0\\
        0&{\color{blue}3}&0\\
        0&0&1
      \end{array}
    \right]
    \left[
      \begin{array}{ccc}
        a_{11}&a_{12}&a_{13}\\
        a_{21}&a_{22}&a_{23}\\
        a_{31}&a_{32}&a_{33}
      \end{array}
    \right]=
    \left[
      \begin{array}{ccc}
        a_{11}&a_{12}&a_{13}\\
        {\color{blue}3}a_{21}&{\color{blue}3}a_{22}&{\color{blue}3}a_{23}\\
        a_{31}&a_{32}&a_{33}
      \end{array}
    \right]
  \end{equation*}\vfill
  and has inverse\vfill
   \begin{equation*}
    \left[
      \begin{array}{ccc}
        1&0&0\\
        0&{\color{blue}\frac{1}{3}}&0\\
        0&0&1
      \end{array}
    \right]
    \left[
      \begin{array}{ccc}
        a_{11}&a_{12}&a_{13}\\
        a_{21}&a_{22}&a_{23}\\
        a_{31}&a_{32}&a_{33}
      \end{array}
    \right]=
    \left[
      \begin{array}{ccc}
        a_{11}&a_{12}&a_{13}\\
        {\color{blue}\frac{1}{3}}a_{21}&{\color{blue}\frac{1}{3}}a_{22}&{\color{blue}\frac{1}{3}}a_{23}\\
        a_{31}&a_{32}&a_{33}
      \end{array}
    \right]
  \end{equation*}\vfill
  that corresponds to multiplying the second row by $\frac{1}{3}$.
\end{frame}

\begin{frame}{Swap two rows}
  The matrix that swaps the second and third rows is:
    \begin{equation*}
    \left[
      \begin{array}{ccc}
        1&0&0\\
        0&0&{\color{blue}1}\\
        0&{\color{red}1}&0\\
      \end{array}
    \right]
    \left[
      \begin{array}{ccc}
        a_{11}&a_{12}&a_{13}\\
        a_{21}&a_{22}&a_{23}\\
        a_{31}&a_{32}&a_{33}
      \end{array}
    \right]=
    \left[
      \begin{array}{ccc}
        a_{11}&a_{12}&a_{13}\\
        {\color{blue}a_{31}}&{\color{blue}a_{32}}&{\color{blue}a_{33}}\\
        {\color{red}a_{21}}&{\color{red}a_{22}}&{\color{red}a_{23}}
      \end{array}
    \right]
  \end{equation*}
  and is its own inverse.
\end{frame}

\begin{frame}{Adding a multiple of another row}
  The matrix that corresponds to subtract $3$ times the first row from the second is:\vfill
  \begin{equation*}\!\!\!\!\!\!\!\!\!
    \left[
      \begin{array}{ccc}
        1&0&0\\
        {\color{blue}-3}&1&0\\
        0&0&1
      \end{array}
    \right]
    \left[
      \begin{array}{ccc}
        a_{11}&a_{12}&a_{13}\\
        a_{21}&a_{22}&a_{23}\\
        a_{31}&a_{32}&a_{33}
      \end{array}
    \right]=
    \left[
      \begin{array}{ccc}
        a_{11}&a_{12}&a_{13}\\
        a_{21}{\color{blue}-3}a_{11}&a_{22}{\color{blue}-3}a_{12}&a_{23}{\color{blue}-3}a_{13}\\
        a_{31}&a_{32}&a_{33}
      \end{array}
    \right]
  \end{equation*}\vfill
  and has inverse\vfill
   \begin{equation*}\!\!\!\!\!\!\!\!\!
    \left[
      \begin{array}{ccc}
        1&0&0\\
        {\color{blue}+3}&1&0\\
        0&0&1
      \end{array}
    \right]
    \left[
      \begin{array}{ccc}
        a_{11}&a_{12}&a_{13}\\
        a_{21}&a_{22}&a_{23}\\
        a_{31}&a_{32}&a_{33}
      \end{array}
    \right]=
    \left[
      \begin{array}{ccc}
        a_{11}&a_{12}&a_{13}\\
        a_{21}{\color{blue}+3}a_{11}&a_{22}{\color{blue}+3}a_{12}&a_{23}{\color{blue}+3}a_{13}\\
        a_{31}&a_{32}&a_{33}
      \end{array}
    \right]
  \end{equation*}
  that corresponds to adding $3$ times the first row to the second.
\end{frame}

\begin{frame}
  Questions?
\end{frame}

\section{Other matrix equations}

\begin{frame}
  \begin{beamercolorbox}[sep=12pt,center]{part title}
    \usebeamerfont{section title}\insertsection\par
  \end{beamercolorbox}
\end{frame}

\begin{frame}{Examples}
  \begin{example}
    Find $A$ such that
    \begin{equation*}
      \left(A+3 \left[
	\begin{array}{ccc}
          1&-1&0\\
          1&2&4
	\end{array}
      \right]\right)^T=
      \left[
	\begin{array}{cc}
          2&1\\
          0&5\\
          3&8
	\end{array}
      \right]
    \end{equation*}
  \end{example}
  \begin{example}
    Find $A$ such that
    \begin{equation*}
      (3I-A^T)^{-1} = 2\left[
	\begin{array}{cc}
          1&1\\
          2&3
	\end{array}
      \right]
    \end{equation*}
  \end{example}
  \begin{example}
    Prove that if $A^3 = 4I$ then $A$ is invertible.
  \end{example}
\end{frame}

\begin{frame}{Examples}
  \begin{example}
    Compute
    \begin{equation*}
      \left[
	\begin{array}{ccc}
          1&0&0\\
          0&1&0\\
          0&0&-2
	\end{array}
      \right]
      \left[
	\begin{array}{cc}
          1&2\\
          3&4\\
          5&6
	\end{array}
      \right]
    \end{equation*}
  \end{example}
  \begin{example}
    Compute
    \begin{equation*}
      \left[
	\begin{array}{cccc}
          1&0&0&0\\
          0&0&0&1\\
          0&0&1&0\\
          0&1&0&0
	\end{array}
      \right]
      \left[
	\begin{array}{ccc}
          1&2&3\\
          4&5&6\\
          7&8&9\\
          10&11&12
	\end{array}
      \right]
    \end{equation*}
  \end{example}
\end{frame}

\begin{frame}
  Questions?
\end{frame}
\end{document}