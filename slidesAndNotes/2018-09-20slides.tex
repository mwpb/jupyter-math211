\documentclass{beamer}

\useoutertheme[subsection=false]{miniframes}
\usecolortheme{beaver}
\setbeamertemplate{navigation symbols}{}
\setbeamertemplate{footline}{}
\usepackage{graphicx}
\usepackage{url}
\usepackage{datetime}

\newcommand {\framedgraphic}[3] {
  \begin{frame}{#1}
    \vspace{-0.5cm}
    \begin{center}
      \includegraphics[width=0.9\textwidth,keepaspectratio]{#2}
    \end{center}
    \vspace{-1cm}
    \begin{center}
      #3
    \end{center}
  \end{frame}
}
\newcommand{\lectureDate}{\formatdate{20}{09}{2018}}

\setbeamertemplate{caption}{\raggedright\insertcaption\par}
\title{MATH211: Linear Methods I}
\author{Matthew Burke}
\date{\lectureDate}
\begin{document}

\frame{\titlepage}

\begin{frame}{Lecture on \lectureDate}
  \tableofcontents
\end{frame}

\section{Last time}
\label{sec:Last-time}

\begin{frame}{Last time}
  \begin{itemize}
  \item (i,j)-entry proofs\vfill
  \item Identity matrices\vfill
  \item Inverse matrices\vfill
  \item Small examples of inverses
  \end{itemize}
\end{frame}

\section{Matrix inversion algorithm}

\begin{frame}
  \begin{beamercolorbox}[sep=12pt,center]{part title}
    \usebeamerfont{section title}\insertsection\par
  \end{beamercolorbox}
\end{frame}

\begin{frame}{Aim of matrix inversion}
  Given $B$ find $A$ such that $BA=I_n$:\vfill
  \begin{equation*}
    \left[
      \begin{array}{cccc}
        B \left[
	\begin{array}{c}
          a_{11}\\
          a_{21}\\
          \vdots\\
          a_{m1}
	\end{array}
        \right]
        B \left[
	\begin{array}{c}
          a_{12}\\
          a_{22}\\
          \vdots\\
          a_{m2}
	\end{array}
        \right]
        \dots
        B \left[
	\begin{array}{c}
          a_{1n}\\
          a_{2n}\\
          \vdots\\
          a_{mn}
	\end{array}
        \right]
      \end{array}
    \right]= \left[
      \begin{array}{cccc}
        1 & 0 & \dots & 0\\
        0 & 1 & \dots & 0\\
        \dots & \dots & \dots & \dots\\
        0 & 0 & \dots & 1
      \end{array}
    \right]
  \end{equation*}\vfill
  I.e. find linear combinations of the columns of $B$ that give each of the standard basis vectors.
\end{frame}

\begin{frame}{The algorithm}
  For the first standard basis vector we solve:
  \begin{equation*}
    Bx = \left[
      \begin{array}{c}
        1\\
        0\\
        \vdots\\
        0
      \end{array}
    \right]\leftrightarrow
    \left[
      \begin{array}{cccc|c}
        b_{11} & b_{12} & \dots & b_{1m} & 1\\
        b_{21} & b_{22} & \dots & b_{2m} & 0\\
        \dots & \dots & \dots & \dots & 0\\
        b_{p1} & b_{p2} & \dots & b_{pm} & 0
      \end{array}
    \right]
  \end{equation*}
  and we can group all together as:
  \begin{equation*}
    \left[
    \begin{array}{cccc|cccc}
      b_{11} & b_{12} & \dots & b_{1m} & 1 & 0 & \dots & 0\\
      b_{21} & b_{22} & \dots & b_{2m} & 0 & 1 & \dots & 0\\
      \dots & \dots & \dots & \dots & \dots & \dots & \dots & \dots\\
      b_{p1} & b_{p2} & \dots & b_{pm} &0 & 0 & \dots & 1
    \end{array}
    \right]
  \end{equation*}
\end{frame}

\begin{frame}{Reduction}
  Row reduce
  \begin{equation*}
    \left[
    \begin{array}{cccc|cccc}
      b_{11} & b_{12} & \dots & b_{1m} & 1 & 0 & \dots & 0\\
      b_{21} & b_{22} & \dots & b_{2m} & 0 & 1 & \dots & 0\\
      \dots & \dots & \dots & \dots & \dots & \dots & \dots & \dots\\
      b_{p1} & b_{p2} & \dots & b_{pm} &0 & 0 & \dots & 1
    \end{array}
    \right]
  \end{equation*}
  to
  \begin{equation*}
    \left[R|A\right]=
    \left[
      \begin{array}{cccc|cccc}
        1 & 0 & \dots & \star & a_{11} & a_{12} &\dots & a_{1n}\\
        0 & 1 & \dots & \star & a_{21} & a_{22} &\dots & a_{2n}\\
        0 & 0 & \dots & \star & \dots & \dots &\dots & \dots\\
        0 & 0 & \dots & \star & a_{m1} & a_{m2} &\dots & a_{mn}
      \end{array}
    \right]
  \end{equation*}
  where $R$ is in reduced row echelon form.
\end{frame}

\begin{frame}{Finishing off}
  \begin{equation*}
    \left[R|A\right]=
    \left[
      \begin{array}{cccc|cccc}
        1 & 0 & \dots & \star & a_{11} & a_{12} &\dots & a_{1n}\\
        0 & 1 & \dots & \star & a_{21} & a_{22} &\dots & a_{2n}\\
        0 & 0 & \dots & \star & \dots & \dots &\dots & \dots\\
        0 & 0 & \dots & \star & a_{m1} & a_{m2} &\dots & a_{mn}
      \end{array}
    \right]
  \end{equation*}\vfill
  If $R$ is:-\vfill
  \begin{description}
  \item [the identity matrix] then $B$ is invertible with inverse $A$\vfill
  \item [anything else] then $B$ is not invertible
  \end{description}
\end{frame}

\begin{frame}{Relationship to solutions of systems}
  Not an efficient method.
\end{frame}

\begin{frame}
  Questions?
\end{frame}

\section{Examples}

\begin{frame}
  \begin{beamercolorbox}[sep=12pt,center]{part title}
    \usebeamerfont{section title}\insertsection\par
  \end{beamercolorbox}
\end{frame}

\begin{frame}{Example}
  
\end{frame}

\begin{frame}
  Questions?
\end{frame}

\section{Properties of inverses}

\begin{frame}
  \begin{beamercolorbox}[sep=12pt,center]{part title}
    \usebeamerfont{section title}\insertsection\par
  \end{beamercolorbox}
\end{frame}

\begin{frame}{Inverse of transpose}
  
\end{frame}

\begin{frame}{Inverses of products}
  
\end{frame}

\begin{frame}{Alternative characterisation of invertible matrices}
  For square matrices $BA=I$ is the same as $AB=I$...
\end{frame}

\section{Other matrix equations}

\begin{frame}
  \begin{beamercolorbox}[sep=12pt,center]{part title}
    \usebeamerfont{section title}\insertsection\par
  \end{beamercolorbox}
\end{frame}

\begin{frame}{Example}
  
\end{frame}

\begin{frame}
  Questions?
\end{frame}

\begin{frame}
  Questions?
\end{frame}


\end{document}