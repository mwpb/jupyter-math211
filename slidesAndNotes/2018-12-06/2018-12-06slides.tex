\documentclass{beamer}

\useoutertheme[subsection=false]{miniframes}
\usecolortheme{beaver}
\setbeamertemplate{navigation symbols}{}
\setbeamertemplate{footline}{}
\usepackage{graphicx}
\usepackage{url}
\usepackage{datetime}
\usepackage{tikz-cd}
\newcommand{\lectureDate}{\formatdate{06}{12}{2018}}

\setbeamertemplate{caption}
{\raggedright\insertcaption\par}
\title{MATH211: Linear Methods I}
\author{Matthew Burke}
\date{\lectureDate}
\begin{document}

\frame{\titlepage}

\begin{frame}{Lecture on \lectureDate}
  \tableofcontents
\end{frame}

\section{Small examples}

\begin{frame}
\begin{beamercolorbox}[sep=12pt,center]{part title}
\usebeamerfont{section title}
\insertsection\par
\end{beamercolorbox}
\end{frame}

\begin{frame}{Examples}
\begin{example}
	Construct an example of a $2\times 2$ matrix with one eigenvalue equal to $3$ that is not diagonal, is not invertible but is diagonalisable.
\end{example}
\begin{example}
	Find a matrix that is not the identity or the zero matrix such that
	\begin{equation*}
	A^2 = A
	\end{equation*}
\end{example}
\end{frame}

\section{Long examples}

\begin{frame}
\begin{beamercolorbox}[sep=12pt,center]{part title}
\usebeamerfont{section title}
\insertsection\par
\end{beamercolorbox}
\end{frame}

\begin{frame}{Google PageRank theory}
Imagine modelling the internet following way:-
\begin{itemize}
	\item Each page consists of links to other pages.
	\item We travel from one page to another by clicking a link at random.
	\item Therefore the probability of going from page $j$ to page $i$ is 
	\begin{equation*}
	\frac{L_{ji}}{Out(j)}
	\end{equation*}
	where $L_{ji}$ is the number of links from page $j$ to page $i$ and $Out(j)$ is the total number of links from page $j$.
	\item A steady state vector for this Markov chain (if it exists) gives the PageRank of the pages.
\end{itemize}
\end{frame}

\begin{frame}{Google PageRank Example}
\begin{example}[Simplified PageRank]
	Suppose that the internet consists of three(!) pages $P_1$, $P_2$ and $P_3$.
	Suppose further that:-
	\begin{itemize}
		\item there are two links from $P_1$ to $P_2$
		\item there is one link from $P_1$ to $P_3$
		\item there is one link from $P_2$ to $P_1$
		\item there is one link from $P_2$ to $P_3$
		\item there is one link from $P_3$ to $P_2$
	\end{itemize}
	Then calculate the (relative) PageRanks for $P_1$, $P_2$ and $P_3$.
\end{example}
\end{frame}

\begin{frame}{Examples}
\begin{example}
Find the distance $d$ between the lines
\begin{equation*}
L_1: \vec{x} = \left[
\begin{matrix}
-6\\
-7\\
7
\end{matrix}
\right]+s
\left[
\begin{matrix}
2\\
1\\
-1
\end{matrix}
\right]\text{ and } L_2: \vec{x} =
\left[
\begin{matrix}
-7\\
-14\\
0
\end{matrix}
\right]+ t \left[
\begin{matrix}
2\\
3\\
-1
\end{matrix}
\right]
\end{equation*}
and find a points $A$ on $L_1$ and $B$ on $L_2$ such that $d(A, B) = d$.
\end{example}
\end{frame}

\begin{frame}{Examples}
\begin{example}
	Suppose that the sequence $x_0$, $x_1$, $x_2$ \dots is defined by $x_0 = 2$, $x_1 = 1+i$ and $x_{k+2} = (1+i)x_{k+1}-ix_k$. Find a formula for $x_k$.
\end{example}
\end{frame}

\begin{frame}{Examples}
\begin{example}
	A draconian country organises its education system as follows:-
	\begin{itemize}
		\item At age $11$ all children take an exam.
		\item The children who pass attend a school of type $A$ and those who fail attend a school of type $B$.
		\item Both schools administer weekly tests.
		\item Consider the fate of a certain individual Adam:-
		\begin{itemize}
			\item Adam has probability $\frac{2}{3}$ of passing the first exam at age $11$.
			\item At school type $A$ Adam has a $\frac{2}{5}$ chance of passing every weekly exam.
			\item At school type $B$ Adam has a $\frac{4}{5}$ chance of passing every weekly exam.
		\end{itemize}
	\end{itemize}
	Model Adam's progress as a Markov chain and find all steady state vectors.
\end{example}
\end{frame}

\begin{frame}{Examples}
\begin{example}
	Find equations for the lines through
	\begin{equation*}
	\left[
	\begin{matrix}
	1\\
	0\\
	1
	\end{matrix}
	\right]
	\end{equation*}
	that meet the line
	\begin{equation*}
	\vec{x} = \left[
	\begin{matrix}
	1\\
	2\\
	0
	\end{matrix}
	\right]+s \left[
	\begin{matrix}
	2\\
	-1\\
	2
	\end{matrix}
	\right]
	\end{equation*}
	at the two points at distance $3$ from 
	\begin{equation*}
	\left[
	\begin{matrix}
	1\\
	2\\
	0
	\end{matrix}
	\right]
	\end{equation*}
\end{example}
\end{frame}

\begin{frame}{Examples}
\begin{example}
	Find the scalar equation for the plane passing through the point
	\begin{equation*}
	\left[
	\begin{matrix}
	5\\
	-5\\
	1
	\end{matrix}
	\right]
	\end{equation*}
	and containing the line
	\begin{equation*}
	\vec{x} = \left[
	\begin{matrix}
	9\\
	-6\\
	-1
	\end{matrix}
	\right]+t \left[
	\begin{matrix}
	-2\\
	-4\\
	5
	\end{matrix}
	\right]
	\end{equation*}
\end{example}
\end{frame}

\begin{frame}{Example}
\begin{example}
If possible diagonalise
\[
\left[
\begin{matrix}
3&-4&2\\
1&-2&2\\
1&-5&5
\end{matrix}
\right]\]
\end{example}
\end{frame}

\section{More small examples}

\begin{frame}
\begin{beamercolorbox}[sep=12pt,center]{part title}
\usebeamerfont{section title}
\insertsection\par
\end{beamercolorbox}
\end{frame}

\begin{frame}{Examples}
\begin{example}
	Compute the orthogonal projection of $u$ onto $v$ where
	\begin{equation*}
	u = \left[
	\begin{matrix}
	-2\\
	-10\\
	-3
	\end{matrix}
	\right]\text{ and } u = \left[
	\begin{matrix}
	3\\
	1\\
	-1
	\end{matrix}
	\right]
	\end{equation*}
\end{example}
\end{frame}


\end{document}