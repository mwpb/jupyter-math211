\documentclass{beamer}

\useoutertheme[subsection=false]{miniframes}
\usecolortheme{beaver}
\setbeamertemplate{navigation symbols}{}
\setbeamertemplate{footline}{}
\usepackage{graphicx}
\usepackage{url}
\usepackage{datetime}

\newcommand {\framedgraphic}[3] {
  \begin{frame}{#1}
    \vspace{-0.5cm}
    \begin{center}
      \includegraphics[width=0.9\textwidth,keepaspectratio]{#2}
    \end{center}
    \vspace{-1cm}
    \begin{center}
      #3
    \end{center}
  \end{frame}
}
\newcommand{\lectureDate}{\formatdate{27}{09}{2018}}

\setbeamertemplate{caption}{\raggedright\insertcaption\par}
\title{MATH211: Linear Methods I}
\author{Matthew Burke}
\date{\lectureDate}
\begin{document}

\frame{\titlepage}

\begin{frame}{Lecture on \lectureDate}
  \tableofcontents
\end{frame}

\section*{Last time}
\label{sec:Last-time}

\begin{frame}{Last time}
  last time
\end{frame}

\section{Determinants}

\begin{frame}
  \begin{beamercolorbox}[sep=12pt,center]{part title}
    \usebeamerfont{section title}\insertsection\par
  \end{beamercolorbox}
\end{frame}

\begin{frame}{Two dimensional determinants}
  First algebra showing get the columns of matrix.
  Then picture in two dimensions.
  The signed area of the parallelogram.
  Calculate.
\end{frame}

\begin{frame}{Effect of scalar multiplication of row}
  Picture.
\end{frame}

\begin{frame}{Determinants of diagonal matrix and scalar multiple}
  Determinant of diagonal.
  Determinant of scalar multiple of a matrix.
\end{frame}

\begin{frame}{Effect of adding a multiple of one row to another}
  
\end{frame}

\begin{frame}{Determinant of triangular matrix}
  We can reduce a triangular matrix to a diagonal one.
\end{frame}

\begin{frame}{Effect of swapping two rows}
  This essentially defines signed area in dimensions higher than three.
\end{frame}

\begin{frame}
  Questions?
\end{frame}

\section{Examples}

\begin{frame}
  \begin{beamercolorbox}[sep=12pt,center]{part title}
    \usebeamerfont{section title}\insertsection\par
  \end{beamercolorbox}
\end{frame}

\begin{frame}{Examples}
  \begin{example}
    Find
    \begin{equation*}
      \left|
	\begin{array}{ccc}
          1&2&3\\
          0&5&6\\
          0&0&9
	\end{array}
      \right|
    \end{equation*}
  \end{example}
  \begin{example}
    Find
    \begin{equation*}
      \left|
        \begin{array}{ccc}
          -3&5&-6\\
          1&-1&3\\
          2&-4&1
        \end{array}
      \right|
    \end{equation*}
  \end{example}
\end{frame}

\begin{frame}{Examples}
  \begin{example}
    Find
    \begin{equation*}
      \left|
        \begin{array}{cccc}
          3&1&2&4\\
          -1&-3&8&0\\
          1&-1&5&5\\
          1&1&2&-1
        \end{array}
      \right|
    \end{equation*}
  \end{example}
  \begin{example}
    If
    \begin{equation*}
      \left|
        \begin{array}{ccc}
          a_1&a_2&a_3\\
          b_1&b_2&b_3\\
          c_1&c_2&c_3
        \end{array}
      \right| = 4 \text{ find }
      \left|
        \begin{array}{ccc}
          -b_1&-b_2&-b_3\\
          a_1+2b_1&a_2+2b_2&a_3+2b_3\\
          3c_1&3c_2&3c_3
        \end{array}
      \right|
    \end{equation*}
  \end{example}
\end{frame}

\begin{frame}{Examples}
  \begin{example}
    Find
    \begin{equation*}
      \left|
        \begin{array}{ccc}
          2&3&5\\
          3&5&9\\
          1&2&4
        \end{array}
      \right|
    \end{equation*}
  \end{example}
\end{frame}

\begin{frame}
  Questions?
\end{frame}

\section{Traditional determinants}

\begin{frame}
  \begin{beamercolorbox}[sep=12pt,center]{part title}
    \usebeamerfont{section title}\insertsection\par
  \end{beamercolorbox}
\end{frame}

\begin{frame}{Minors and cofactors}
  Definitions. Slide 6 of general notes.
\end{frame}

\begin{frame}{Recursive definition of determinant}
  In terms of cofactors.
\end{frame}


\begin{frame}
  Questions?
\end{frame}

\section{Examples}

\begin{frame}
  \begin{beamercolorbox}[sep=12pt,center]{part title}
    \usebeamerfont{section title}\insertsection\par
  \end{beamercolorbox}
\end{frame}

\begin{frame}{Examples}
  \begin{example}
    Find the (1,2)-minor of
    \begin{equation*}
      \left[
	\begin{array}{ccc}
          1&1&3\\
          2&4&1\\
          5&2&6
	\end{array}
      \right]
    \end{equation*}
  \end{example}
  \begin{example}
    Find
    \begin{equation*}
      \left|
	\begin{array}{ccc}
          1&1&3\\
          2&4&1\\
          5&2&6
	\end{array}
      \right|
    \end{equation*}
  \end{example}
\end{frame}

\begin{frame}{Examples}
  \begin{example}
    Find
    \begin{equation*}
      \left|
        \begin{array}{cccc}
          0&1&-2&1\\
          5&0&0&7\\
          0&1&-1&0\\
          3&0&0&2
        \end{array}
      \right|
    \end{equation*}
  \end{example}
  \begin{example}
    Find
    \begin{equation*}
      \left|
        \begin{array}{cccc}
          -8&1&0&-4\\
          5&7&0&-7\\
          12&-3&0&8\\
          -3&11&0&2
        \end{array}
      \right|
    \end{equation*}
  \end{example}
\end{frame}

\begin{frame}
  Questions?
\end{frame}

\end{document}