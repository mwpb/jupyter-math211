\documentclass{beamer}

\useoutertheme[subsection=false]{miniframes}
\usecolortheme{beaver}
\setbeamertemplate{navigation symbols}{}
\setbeamertemplate{footline}{}
\usepackage{graphicx}
\usepackage{url}
\usepackage{datetime}
\usepackage{tikz-cd}
\newcommand{\lectureDate}{\formatdate{06}{11}{2018}}

\setbeamertemplate{caption}
{\raggedright\insertcaption\par}
\title{MATH211: Linear Methods I}
\author{Matthew Burke}
\date{\lectureDate}
\begin{document}

\frame{\titlepage}

\begin{frame}{Lecture on \lectureDate}
  \tableofcontents
\end{frame}

\section*{Last time}
\label{sec:Last-time}

\begin{frame}{Last time}
  \begin{itemize}
  \item Matrices and linear transformations\vfill
  \item Composition of linear transformations\vfill
  \item Inverse of linear transformations
  \end{itemize}
\end{frame}

\section{Examples}

\begin{frame}
\begin{beamercolorbox}[sep=12pt,center]{part title}
\usebeamerfont{section title}
\insertsection\par
\end{beamercolorbox}
\end{frame}

\begin{frame}{Examples}
\begin{example}
If
\begin{equation*}
S \left[
\begin{array}{c}
x\\
y
\end{array}
\right] = \left[
\begin{array}{c}
x\\
-y
\end{array}
\right]\text{ and } T \left[
\begin{array}{c}
x\\
y
\end{array}
\right] = \left[
\begin{array}{c}
-y\\
x
\end{array}
\right]
\end{equation*}
then find $S\circ T$, $T\circ S$ and the matrices $[S\circ T]$ and $[T\circ S]$.
\end{example}

\begin{example}
If
\begin{equation*}
T \left[
\begin{array}{c}
x\\
y
\end{array}
\right] = \left[
\begin{array}{c}
x+y\\
y
\end{array}
\right]
\end{equation*}
find $T^{-1}$ and $[T^{-1}]$.
\end{example}
\end{frame}


\section{Complex numbers}

\begin{frame}
\begin{beamercolorbox}[sep=12pt,center]{part title}
\usebeamerfont{section title}
\insertsection\par
\end{beamercolorbox}
\end{frame}

\begin{frame}{Progression of thought about complex numbers}

% \begin{quote}
% .. subtle as they are useless ..
% \end{quote}
% (Girolamo Cardano - discoverer of complex numbers)
% \vfill
\begin{quote}
\dots the whole matter seems to rest on sophistry rather than truth\dots
\end{quote}
(Rafael Bombelli 1526-1572. Early innovator in use of complex numbers)
\vfill
\begin{quote}
The shortest path between two truths in the real domain passes through the complex domain.
\end{quote}
(Jacques Hadamard 1865-1963)
% \begin{quote}
% Indeed, nowadays no electrical engineer could get along without complex numbers, and neither could anyone working in aerodynamics or fluid dynamics.
% \end{quote}
% (Keith Devlin - British mathematician)

% \vfill

% \begin{quote}
% The imaginary number is a fine and wonderful resource of the human spirit, almost an amphibian between being and not being.
% \end{quote}
% (Leibniz)
\end{frame}

% \begin{frame}{More recently...}
% \begin{quote}
% It has been written that the shortest and best way between two truths of the real domain often passes through the imaginary one.
% \end{quote}
% (Jacques Hadamard)
% \vfill

% \begin{quote}
% I tell you, with complex numbers you can do anything.
% \end{quote}
% (John Derbyshire - mathematics writer)
% \vfill

% \begin{quote}
% Indeed, nowadays no electrical engineer could get along without complex numbers, and neither could anyone working in aerodynamics or fluid dynamics.
% \end{quote}
% (Keith Devlin - British mathematician)
% \end{frame}

\begin{frame}{Algebraic motivation - extending the number system}
Suppose that we only knew about the natural numbers:
\begin{equation*}
\mathbb{N} := \{0, 1, 2, 3, \dots\}.
\end{equation*}
Then we wouldn't have a solution to the equation:
\begin{equation*}
 x + 1 = 0
\end{equation*}
{\bf Solution:}
\begin{itemize}
	\item Add a new symbol $-1$ such that
	\begin{equation*}
	(-1)+1 = 0
	\end{equation*}
	\item Figure out addition and multiplication for this new symbol.
	\item (So in particular forced to add $-k$ for all $k\in \mathbb{N}$.)
\end{itemize}
\end{frame}

\begin{frame}{Algebraic motivation - extending the number system}
\begin{quote}
God made the integers; all else is the work of man.
\end{quote}
(Leopold Kronecker 1823-1891)\vfill
\begin{itemize}
	\item Using only positive integers $\{1, 2, 3, \dots\}$
	\begin{itemize}
		\item we cannot find a solution to $x+1 = 0$.
	\end{itemize}\vfill
	\item Using only integers $\{\dots, -2, -1, 0, 1, 2, \dots\}$
	\begin{itemize}
		\item we cannot find a solution to $3x+2 = 0$.
	\end{itemize}\vfill
	\item Using only fractions $\{\frac{a}{b} \text{ where } b\neq0 \}$
	\begin{itemize}
		\item we cannot find a solution to $x^2-2 =0$.
	\end{itemize}\vfill
	\item Using only real numbers (decimals)
	\begin{itemize}
		\item we cannot find a solution to $x^2 + 1 = 0$\dots
	\end{itemize}
\end{itemize}
\end{frame}

\begin{frame}{Algebraic motivation - extending the number system}
Suppose that we only knew about real numbers.\vfill
Then we wouldn't have a solution to the equation:
\begin{equation*}
 x^2 + 1 = 0
\end{equation*}
{\bf Solution:}
\begin{itemize}
	\item Add a new symbol $i$ such that
	\begin{equation*}
	i^2 = -1
	\end{equation*}
	\item Figure out addition and multiplication for this new symbol.
	\begin{itemize}
		\item (Later in this lecture.)
	\end{itemize}
	\item It turns out that we are forced to add all numbers of the form:
	\begin{equation*}
	a+ib
	\end{equation*}
	where $a$ and $b$ are real numbers.
\end{itemize}
\end{frame}

\begin{frame}{Complex numbers}
\begin{definition}
\begin{itemize}
\item
The \emph{imaginary unit}, denoted $i$, is defined to be
a number with the property that $i^2=-1$.
\item
A \emph{pure imaginary} number has the form $bi$ where 
$b\in \mathbb{R}$, $b\neq 0$.
\item
A \emph{complex number} is any number $z$ of the form
\[ z = a + bi\]
where $a,b\in \mathbb{R}$ and $i$ is the imaginary unit.
\end{itemize}
\end{definition}
\begin{definition}
If $z = a + bi$ then:
\begin{itemize}
\item
$a$ is called the \emph{real part} of $z$.
\item
$b$ is called the \emph{imaginary part} of $z$.
\end{itemize}
\end{definition}
\end{frame}

\begin{frame}{Questions}
Questions?
\end{frame}

\section{Plus and times}

\begin{frame}
\begin{beamercolorbox}[sep=12pt,center]{part title}
\usebeamerfont{section title}
\insertsection\par
\end{beamercolorbox}
\end{frame}

\begin{frame}{Motivation}
Treat $i$ as a number that satisfies the usual laws. E.g.:-
\begin{itemize}
	\item $a(b+c) = ab+ac$
	\item $a+b = b+a$
	\item $ab = ba$ \dots etc \dots
\end{itemize}
as well as
\begin{equation*}
i^2=-1
\end{equation*}
and see what happens.
\begin{example}
\begin{align*}
1+i+i^2+i^3+i^4+i^5 &= 1+i-1-i+1+i \\
&= 1+i
\end{align*}
which cannot be simplified any further.\\
(Because we only know $i^2=-1$.)
\end{example}
\end{frame}

\begin{frame}{Addition}
\begin{example}
We would want addition to satisfy:-
\begin{itemize}
	\item $i+i = 2i$
	\item $i-i = 0$
	\item $(1+i)+2 = 3+i$
\end{itemize}
\end{example}
\begin{definition}
If $a$, $b$, $c$ and $d$ are real numbers then:
	\begin{equation*}
	(a+bi)+(c+di) = (a+c)+(b+d)i
	\end{equation*}
\end{definition}
\end{frame}

\begin{frame}{Multiplication}
\begin{example}
We would want multiplication to satisfy:-
	\begin{itemize}
		\item $i\cdot i = -1$
		\item $2\cdot i = 2i$
		\item $(1+i)\cdot i = i+i\cdot i = -1+i$
	\end{itemize}
\end{example}
\begin{definition}
If $a$, $b$, $c$ and $d$ are real numbers then
\begin{equation*}
(a+bi)\cdot(c+di) = ac+adi+bci-db = (ac-bd)+(ad+bc)i
\end{equation*}
\end{definition}
\end{frame}


\begin{frame}{Examples}
\begin{example}
Evaluate
\begin{itemize}
\item $(-3+6i) + (5-i)$.
% Ans: 2+5i
\item $(4-7i) + (6-2i)$.
% 10-9i
\item $(-3+6i) - (5-i)$.
%-8+7i
\item $(4-7i) - (6-2i)$.
%-2-5i
\end{itemize}
\end{example}
\begin{example}
\begin{equation*} 
(2-3i)(-3+4i) 
\end{equation*}
% ANS: 6+17i
\end{example}
\begin{example}
Find all complex numbers $z$ such that $z^2 = -3+4i$.
%% Ans: z = 1+2i or z = -1-2i
\end{example}
\end{frame}

\begin{frame}{Equations}
\begin{definition}
If $a$, $b$, $c$ and $d$ are real numbers then
\begin{equation*}
a+bi = c+di
\end{equation*}
if and only if $a=c$ and $b=d.$
\end{definition}
\end{frame}

\begin{frame}{Questions?}
Questions?
\end{frame}

\section{Graphical interpretation}

\begin{frame}
\begin{beamercolorbox}[sep=12pt,center]{part title}
\usebeamerfont{section title}
\insertsection\par
\end{beamercolorbox}
\end{frame}

\begin{frame}[fragile]{Graphical interpretation}
Consider the function $\mathbb{R} \rightarrow \mathbb{R}$ that multiplies by $-1$:
\begin{equation*}
\begin{tikzcd}
-3\arrow[bend left, color = blue]{rrrrrr} & -2\arrow[bend left, color = blue]{rrrr} & -1 \arrow[bend left, color = blue]{rr} & 0 & +1\arrow[bend left, color = blue]{ll} & +2\arrow[bend left, color = blue]{llll} & +3\arrow[bend left, color = blue]{llllll}\\
\end{tikzcd}
\end{equation*}
which shows us that $-1\times(-1\times x) = x$.
\end{frame}

\begin{frame}[fragile]{Graphical interpretation}
So how do we get a transformation that squares to $-1$?
\begin{equation*}
\begin{tikzcd}
{} & {} & +i \arrow[bend right, color = blue]{dl} & {} & {}\\
-2 & -1\arrow[bend right, color = blue]{dr} & 0 & +1 \arrow[bend right, color = blue]{ul} & +2\\
{} & {} & -i\arrow[bend right, color = blue]{ur} & {} & {}
\end{tikzcd}
\end{equation*}
\end{frame}

\begin{frame}[fragile]{Graphical interpretation}
So we draw the complex number $a+bi$ as the point $(a, b)$ in $\mathbb{R}^2$:
\includegraphics{complex-plane.png}
\end{frame}

\begin{frame}{Addition of complex numbers}
\begin{columns}
\column{0.4\textwidth}
\includegraphics[scale=0.5]{complex-addition.png}
\column{0.5\textwidth}
Complex addition is defined component-wise:
\begin{equation*}
(a+ib)+(c+id) = (a+c)+(b+d)i
\end{equation*}
and therefore has the same graphical interpretation as the addition of vectors in $\mathbb{R}^2$.
\end{columns}
\end{frame}


\begin{frame}{Historical curiosity}
Apparently Bombelli describes complex arithmetic thus:
\begin{quote}
"Plus by plus of minus, makes plus of minus.
Minus by plus of minus, makes minus of minus.
Plus by minus of minus, makes minus of minus.
Minus by minus of minus, makes plus of minus.
Plus of minus by plus of minus, makes minus.
Plus of minus by minus of minus, makes plus.
Minus of minus by plus of minus, makes plus.
Minus of minus by minus of minus makes minus." 
\end{quote}
according to Crossley in `The Emergence of Number'.
A hint from Crossley is: the last two lines would become 
\begin{equation*}
-\sqrt{-a}\cdot\sqrt{-b}= \sqrt{ab}\text{ and }(-\sqrt{-a})\cdot(-\sqrt{-b})= -\sqrt{ab}
\end{equation*}
in modern notation.
\end{frame}

\begin{frame}{Summary}
\begin{itemize}
	\item Graphical interpretation.
	\item Addition: algebra and geometry.
	\item Multiplication: algebra and geometry.
	\item Conjugate: algebra and geometry.
\end{itemize}
\end{frame}

\begin{frame}{Clearing denominators}
\begin{example}
\begin{itemize}
\item If $z=3+4i$, then $\overline{z}= 3-4i$,
i.e., $\overline{3+4i}=3-4i$.
\item $\overline{-2+5i}= -2-5i$.
\item $\overline{i}= -i$.
\item $\overline{7}= 7$.
\end{itemize}
\end{example}
\begin{example}[Exemplar]
Write 
\begin{equation*}
\frac{a+bi}{c+di}
\end{equation*}
in the form $e+fi$ for real numbers $a$, $b$, $c$, $d$, $e$, and $f$.
\end{example}
\begin{itemize}
\item $\frac{1}{i}$ % ANS -i
\item $\frac{2-i}{3+4i}$ % ANS 2/25 -(11/25)i
\item $\frac{1-2i}{-2+5i}$ % ANS -(12/29) -(1/29)i
\end{itemize}
\end{frame}
% Stopped at page 17 of first set of slides.


\end{document}