\documentclass{beamer}

\useoutertheme[subsection=false]{miniframes}
\usecolortheme{beaver}
\setbeamertemplate{navigation symbols}{}
\setbeamertemplate{footline}{}
\usepackage{graphicx}
\usepackage{url}
\usepackage{datetime}

\newcommand {\framedgraphic}[3] {
  \begin{frame}{#1}
    \vspace{-0.5cm}
    \begin{center}
      \includegraphics[width=0.9\textwidth,keepaspectratio]{#2}
    \end{center}
    \vspace{-1cm}
    \begin{center}
      #3
    \end{center}
  \end{frame}
}
\newcommand{\lectureDate}{\formatdate{25}{09}{2018}}

\setbeamertemplate{caption}{\raggedright\insertcaption\par}
\title{MATH211: Linear Methods I}
\author{Matthew Burke}
\date{\lectureDate}
\begin{document}

\frame{\titlepage}

\begin{frame}{Lecture on \lectureDate}
  \tableofcontents
\end{frame}

\section*{Last time}
\label{sec:Last-time}

\begin{frame}{Last time}
  \begin{itemize}
  \item Matrix inversion algorithm\vfill
  \item Examples\vfill
  \item Inverses of products and transposes
  \end{itemize}
\end{frame}

\begin{frame}{Some applications of inverse matrices}
  \begin{itemize}
  \item Stochastic systems represented by matrix $A$
    \begin{itemize}
    \item $a_{ij}$ denotes the probability of transition from state $i$ to state $j$
    \item sometimes after many iterations these systems
      \begin{itemize}
      \item reach a `steady state'
      \item or become constrained to a certain set of states
      \end{itemize}
    \item being invertible implies that all states are always possible
    \end{itemize}\vfill
  \item The inverse function theorem
    \begin{itemize}
    \item if a function has invertible derivative at a point
    \item then it is invertible in some neighbourhood of the point
    \item the domain and the codomain are `locally the same'
    \end{itemize}
  \end{itemize}
\end{frame}

\section{Elementary matrices}

\begin{frame}
  \begin{beamercolorbox}[sep=12pt,center]{part title}
    \usebeamerfont{section title}\insertsection\par
  \end{beamercolorbox}
\end{frame}


\begin{frame}{Elementary matrices}
  Recall the elementary row operations:\vfill
  \begin{enumerate}
  \item Swap two rows.
  \item Multiply a row by a non-zero scalar.
  \item Add a multiple of one row to another.
  \end{enumerate}\vfill
  Each corresponds to multiplying on the left\vfill
  \begin{equation*}
    A \mapsto EA
  \end{equation*}\vfill
  by a different \emph{elementary matrix $E$}.
\end{frame}

\begin{frame}{Multiply a row by a non-zero scalar}
  The matrix that corresponds to multiplying the second row by $3$ is:\vfill
  \begin{equation*}
    \left[
      \begin{array}{ccc}
        1&0&0\\
        0&{\color{blue}3}&0\\
        0&0&1
      \end{array}
    \right]
    \left[
      \begin{array}{ccc}
        a_{11}&a_{12}&a_{13}\\
        a_{21}&a_{22}&a_{23}\\
        a_{31}&a_{32}&a_{33}
      \end{array}
    \right]=
    \left[
      \begin{array}{ccc}
        a_{11}&a_{12}&a_{13}\\
        {\color{blue}3}a_{21}&{\color{blue}3}a_{22}&{\color{blue}3}a_{23}\\
        a_{31}&a_{32}&a_{33}
      \end{array}
    \right]
  \end{equation*}\vfill
  and has inverse\vfill
   \begin{equation*}
    \left[
      \begin{array}{ccc}
        1&0&0\\
        0&{\color{blue}\frac{1}{3}}&0\\
        0&0&1
      \end{array}
    \right]
    \left[
      \begin{array}{ccc}
        a_{11}&a_{12}&a_{13}\\
        a_{21}&a_{22}&a_{23}\\
        a_{31}&a_{32}&a_{33}
      \end{array}
    \right]=
    \left[
      \begin{array}{ccc}
        a_{11}&a_{12}&a_{13}\\
        {\color{blue}\frac{1}{3}}a_{21}&{\color{blue}\frac{1}{3}}a_{22}&{\color{blue}\frac{1}{3}}a_{23}\\
        a_{31}&a_{32}&a_{33}
      \end{array}
    \right]
  \end{equation*}\vfill
  that corresponds to multiplying the second row by $\frac{1}{3}$.
\end{frame}

\begin{frame}{Swap two rows}
  The matrix that swaps the second and third rows is:
    \begin{equation*}
    \left[
      \begin{array}{ccc}
        1&0&0\\
        0&0&{\color{blue}1}\\
        0&{\color{red}1}&0\\
      \end{array}
    \right]
    \left[
      \begin{array}{ccc}
        a_{11}&a_{12}&a_{13}\\
        a_{21}&a_{22}&a_{23}\\
        a_{31}&a_{32}&a_{33}
      \end{array}
    \right]=
    \left[
      \begin{array}{ccc}
        a_{11}&a_{12}&a_{13}\\
        {\color{blue}a_{31}}&{\color{blue}a_{32}}&{\color{blue}a_{33}}\\
        {\color{red}a_{21}}&{\color{red}a_{22}}&{\color{red}a_{23}}
      \end{array}
    \right]
  \end{equation*}
  and is its own inverse.
\end{frame}

\begin{frame}{Adding a multiple of another row}
  The matrix that corresponds to subtract $3$ times the first row from the second is:
  \begin{equation*}\!\!\!\!\!\!\!\!\!
    \left[
      \begin{array}{ccc}
        1&0&0\\
        {\color{blue}-3}&1&0\\
        0&0&1
      \end{array}
    \right]
    \left[
      \begin{array}{ccc}
        a_{11}&a_{12}&a_{13}\\
        a_{21}&a_{22}&a_{23}\\
        a_{31}&a_{32}&a_{33}
      \end{array}
    \right]=
    \left[
      \begin{array}{ccc}
        a_{11}&a_{12}&a_{13}\\
        a_{21}{\color{blue}-3}a_{11}&a_{22}{\color{blue}-3}a_{12}&a_{23}{\color{blue}-3}a_{13}\\
        a_{31}&a_{32}&a_{33}
      \end{array}
    \right]
  \end{equation*}\vfill
  and has inverse
   \begin{equation*}\!\!\!\!\!\!\!\!\!
    \left[
      \begin{array}{ccc}
        1&0&0\\
        {\color{blue}+3}&1&0\\
        0&0&1
      \end{array}
    \right]
    \left[
      \begin{array}{ccc}
        a_{11}&a_{12}&a_{13}\\
        a_{21}&a_{22}&a_{23}\\
        a_{31}&a_{32}&a_{33}
      \end{array}
    \right]=
    \left[
      \begin{array}{ccc}
        a_{11}&a_{12}&a_{13}\\
        a_{21}{\color{blue}+3}a_{11}&a_{22}{\color{blue}+3}a_{12}&a_{23}{\color{blue}+3}a_{13}\\
        a_{31}&a_{32}&a_{33}
      \end{array}
    \right]
  \end{equation*}\vfill
  that corresponds to adding $3$ times the first row to the second.
\end{frame}

\begin{frame}{How to find the corresponding matrix?}
  Suppose we have an elementary row operation
  \begin{equation*}
    A\mapsto e(A)
  \end{equation*}
  We know that there is some $E$ such that
  \begin{equation*}
    e(A) = EA
  \end{equation*}
  therefore to find $E$ we can let $A=I$:
  \begin{equation*}
    e(I) = EI = E
  \end{equation*}
  \bf{In words: apply the row operation to the identity matrix.}
\end{frame}

\begin{frame}
  Questions?
\end{frame}

\section{Examples}

\begin{frame}
  \begin{beamercolorbox}[sep=12pt,center]{part title}
    \usebeamerfont{section title}\insertsection\par
  \end{beamercolorbox}
\end{frame}

\begin{frame}{Examples}
  \begin{example}
    Compute
    \begin{equation*}
      \left[
	\begin{array}{ccc}
          1&0&0\\
          0&1&0\\
          0&0&-2
	\end{array}
      \right]
      \left[
	\begin{array}{cc}
          1&2\\
          3&4\\
          5&6
	\end{array}
      \right]
    \end{equation*}
  \end{example}
  \begin{example}
    Compute
    \begin{equation*}
      \left[
	\begin{array}{cccc}
          1&0&0&0\\
          0&0&0&1\\
          0&0&1&0\\
          0&1&0&0
	\end{array}
      \right]
      \left[
	\begin{array}{ccc}
          1&2&3\\
          4&5&6\\
          7&8&9\\
          10&11&12
	\end{array}
      \right]
    \end{equation*}
  \end{example}
\end{frame}

\begin{frame}{Example}
  \begin{example}
    If
    \begin{equation*}
      A = \left[
        \begin{array}{cc}
          4&1\\
          1&3
        \end{array}
      \right]\text{ and }
      C= \left[
        \begin{array}{cc}
          1&3\\
          2&-5
        \end{array}
      \right]
    \end{equation*}
    then find elementary matrices such that $C = FEA$.
  \end{example}
  \begin{example}
    Write
    \begin{equation*}
      \left[
	\begin{array}{ccc}
          1&2&-4\\
          -3&-6&13\\
          0&-1&2
	\end{array}
      \right]
    \end{equation*}
    as a product of elementary matrices.
  \end{example}
\end{frame}

\begin{frame}{Examples}
  \begin{example}
    Write
    \begin{equation*}
      \left[
	\begin{array}{cc}
          4&1\\
          -3&2
	\end{array}
      \right]
    \end{equation*}
    as a product of elementary matrices.
  \end{example}
  \begin{example}
    Find a matrix $U$ such that
    \begin{equation*}
      U
      \left[
	\begin{array}{ccc}
          3&0&1\\
          2&-1&0
	\end{array}
      \right]
    \end{equation*}
    is in reduced row echelon form.
  \end{example}
\end{frame}

\begin{frame}
  Questions?
\end{frame}

\section{Some theory}

\begin{frame}
  \begin{beamercolorbox}[sep=12pt,center]{part title}
    \usebeamerfont{section title}\insertsection\par
  \end{beamercolorbox}
\end{frame}

\begin{frame}{Row reduction as factorisation}
  \begin{itemize}
  \item If we row reduce
    \begin{equation*}
      A \mapsto R
    \end{equation*}
    then there is a sequence of elementary matrices $E_i$ such that
    \begin{equation*}
      E_n\dots E_2E_1A = R
    \end{equation*}
  \item In augmented matrix notation
  \begin{equation*}
    [A|I]\mapsto [R|E_n\dots E_2E_1]
  \end{equation*}
  because the row operations that reduce $A\mapsto R$ are precisely those that take $I\mapsto E_n\dots E_2E_1$.
  \end{itemize}
\end{frame}

\begin{frame}{Invertibility of square matrices}
  \begin{theorem}[The inversion algorithm works]
    If $B$ is a square matrix and $[B|I]$ row reduces as
    \begin{equation*}
      [B|I] \mapsto [I|A]
    \end{equation*}
    then $BA=I$ and $AB=I$.    
    \begin{proof}
      If $[B|I]\mapsto [I|A]$ then there is a sequence of elementary matrices $E_i$ such that
      \begin{equation*}
        I = E_n\dots E_2E_1B
      \end{equation*}
      and
      \begin{equation*}
        A = E_n\dots E_2E_1I
      \end{equation*}
      so $AB=I$ and $BA = (E_n\dots E_2E_1)^{-1}(E_n\dots E_2E_1) = I$ because the elementary matrices $E_i$ are invertible.
    \end{proof}
  \end{theorem}
\end{frame}

\begin{frame}{Uniqueness of reduced row echelon form}
  \begin{theorem}
    If $R$ and $S$ are reduced row echelon matrices and there is a sequence of elementary row operations converting $R$ into $S$ then $R=S$.
  \end{theorem}\vfill
  (So the reduced row echelon form is unique.)
\end{frame}

\begin{frame}{Alternative characterisation of invertible matrices}
  \begin{theorem}
    Let $A$ be an $n\times n$ matrix, and
    let $X$, $B$ be $n \times 1$ vectors.
    The following conditions are equivalent.
    \begin{enumerate}
    \item $A$ is invertible.
    \item The rank of $A$ is $n$. 
    \item The reduced row echelon form of $A$ is $I_n$.
    \item $AX=0$ has only the trivial solution, $X=0$.
    \item $A$ can be transformed to $I_n$ by elementary row operations.
    \item
      The system $AX=B$ has a unique solution $X$ for any
      choice of $B$.
    \item
      There exists an $n\times n$ matrix $C$ with the property that $CA=I_n$.
    \item
      There exists an $n\times n$ matrix $C$ with the property that $AC=I_n$.
    \end{enumerate}
  \end{theorem}
\end{frame}

\begin{frame}
  Questions?
\end{frame}

\section{Other matrix equations}

\begin{frame}
  \begin{beamercolorbox}[sep=12pt,center]{part title}
    \usebeamerfont{section title}\insertsection\par
  \end{beamercolorbox}
\end{frame}

\begin{frame}{Examples}
  \begin{example}
    Find $A$ such that
    \begin{equation*}
      \left(A+3 \left[
	\begin{array}{ccc}
          1&-1&0\\
          1&2&4
	\end{array}
      \right]\right)^T=
      \left[
	\begin{array}{cc}
          2&1\\
          0&5\\
          3&8
	\end{array}
      \right]
    \end{equation*}
  \end{example}
  \begin{example}
    Find $A$ such that
    \begin{equation*}
      (3I-A^T)^{-1} = 2\left[
	\begin{array}{cc}
          1&1\\
          2&3
	\end{array}
      \right]
    \end{equation*}
  \end{example}
  \begin{example}
    Prove that if $A^3 = 4I$ then $A$ is invertible.
  \end{example}
\end{frame}

\begin{frame}{Examples}
  \begin{example}
    Which of the following equations is true for arbitrary matrices $A$ and $B$?
    \begin{enumerate}
    \item $(A-B)^2 = A^2-2AB+B^2$
    \item $(A+B)(A-B) = A^2-B^2$
    \item $A^2B^2 = A(AB)B$
    \end{enumerate}
  \end{example}
\end{frame}

\begin{frame}{Example}
  \begin{example}
    For what values of $a$ does the system
    \begin{equation*}
      \left[
	\begin{array}{ccc|c}
          1&2&-1&-4\\
          -2&-3&2&4\\
          a&-5&1&16
	\end{array}
      \right]
    \end{equation*}
    have
    \begin{itemize}
    \item infinitely many solutions
    \item a unique solution
    \item no solutions.
    \end{itemize}
  \end{example}
\end{frame}

\begin{frame}
  Questions?
\end{frame}

\end{document}