\documentclass{beamer}

\useoutertheme[subsection=false]{miniframes}
\usecolortheme{beaver}
\setbeamertemplate{navigation symbols}{}
\setbeamertemplate{footline}{}
\usepackage{graphicx}
\usepackage{url}
\usepackage{datetime}
\usepackage{tikz-cd}
\newcommand{\lectureDate}{\formatdate{25}{10}{2018}}

\setbeamertemplate{caption}
{\raggedright\insertcaption\par}
\title{MATH211: Linear Methods I}
\author{Matthew Burke}
\date{\lectureDate}
\begin{document}

\frame{\titlepage}

\begin{frame}{Lecture on \lectureDate}
  \tableofcontents
\end{frame}

\section*{Last time}
\label{sec:Last-time}

\begin{frame}{Last time}
  \begin{itemize}
  \item Midterm review\vfill
  \item Cross product\vfill
  \item Applications to distance, area and volume\vfill
  \end{itemize}
\end{frame}

\section{Misc}

\begin{frame}
\begin{beamercolorbox}[sep=12pt,center]{part title}
\usebeamerfont{section title}
\insertsection\par
\end{beamercolorbox}
\end{frame}

\begin{frame}{Triangle inequality}
\begin{theorem}
  If $u$ and $v$ are vectors in $\mathbb{R}^n$ then
  \begin{equation*}
    \|u+v\| \leq \|u\|+\|v\|
  \end{equation*}
\end{theorem}\vfill
\begin{theorem}
  If $u$ and $v$ are vectors in $\mathbb{R}^n$ then
  \begin{equation*}
    |\|u\|-\|v\||\leq \|u-v\|
  \end{equation*}
\end{theorem}
\end{frame}

\begin{frame}{The Lagrange identity}
The following identity follows directly from the Pythagorean theorem.
\begin{theorem}
If $u$ and $v$ are in $\mathbb{R}^3$ then
\begin{equation*}
\|u\|^2\|v\|^2 = \|u\times v\|^2 + (u\cdot v)^2
\end{equation*}
\end{theorem}
\end{frame}

\section{Examples}

\begin{frame}{Example}
    \begin{example}
        Find equations for the lines through
        \begin{equation*}
            \left[
            \begin{array}{c}
            1\\
            0\\
            1
            \end{array}
            \right]
        \end{equation*}
        that meet the line
        \begin{equation*}
            \left[
            \begin{array}{c}
            x\\
            y\\
            z
            \end{array}
            \right] = \left[
            \begin{array}{c}
            1\\
            2\\
            0
            \end{array}
            \right]+s \left[
            \begin{array}{c}
            2\\
            -1\\
            2
            \end{array}
            \right]
        \end{equation*}
        at the two points distance three from $(1, 2, 0)$.
    \end{example}
\end{frame}

\begin{frame}{Examples}
\begin{example}
Find the area of the triangle having vertices
\begin{equation*}
\left[
\begin{array}{c}
3\\
-1\\
2
\end{array}
\right], \left[
\begin{array}{c}
1\\
1\\
0
\end{array}
\right]\text{ and } \left[
\begin{array}{c}
1\\
2\\
-1
\end{array}
\right]
\end{equation*}
\end{example}
\begin{example}
Find the volume of the parallelepiped determined by the vectors
\begin{equation*}
\left[
\begin{array}{c}
2\\
1\\
-1
\end{array}
\right], \left[
\begin{array}{c}
1\\
0\\
2
\end{array}
\right]\text{ and } \left[
\begin{array}{c}
2\\
1\\
1
\end{array}
\right]
\end{equation*}
\end{example}
\end{frame}

\begin{frame}{Examples}
    \begin{example}
        The diagonals of a parallelogram bisect each other.
      \end{example}
      \begin{example}
        If $ABCD$ is an arbitrary quadrilateral then the midpoints of the four sides form a parallelogram.
      \end{example}
\end{frame}

\begin{frame}{Examples}
\begin{example}
Find the shortest distance between the following plane and line.
\begin{equation*}
X_1:x-y+2z=2, L_1:\left[
\begin{array}{c}
1\\
1\\
3
\end{array}
\right]+t \left[
\begin{array}{c}
1\\
-1\\
-1
\end{array}
\right]
\end{equation*}
% Ans: 2/3 sqrt(6)
\end{example}
\begin{example}
Find the shortest distance between the following two planes.
\begin{equation*}
X_1:x-y+2z=5, X_2:2x-2y+4z=4
\end{equation*}
% Ans: (1/2) sqrt(6)
\end{example}
\end{frame}

\begin{frame}{Examples}
\begin{example}
Find the distance between the following two lines.
\begin{equation*}
\left[
\begin{array}{c}
x\\
y\\
z
\end{array}
\right] = \left[
\begin{array}{c}
3\\
1\\
-1
\end{array}
\right]+s \left[
\begin{array}{c}
1\\
1\\
-1
\end{array}
\right]\text{ and } \left[
\begin{array}{c}
x\\
y\\
z
\end{array}
\right] = \left[
\begin{array}{c}
1\\
2\\
0
\end{array}
\right] +t \left[
\begin{array}{c}
1\\
0\\
2
\end{array}
\right]
\end{equation*}
% Ans: (4/7) sqrt(14)
\end{example}
\end{frame}

\section{Linear transformations}

\begin{frame}
\begin{beamercolorbox}[sep=12pt,center]{part title}
\usebeamerfont{section title}
\insertsection\par
\end{beamercolorbox}
\end{frame}

\begin{frame}{Functions between Euclidean spaces}
\begin{definition}
A \emph{function $f$ from $\mathbb{R}^n$ to $\mathbb{R}^m$ written}
\begin{equation*}
f:\mathbb{R}^n \rightarrow \mathbb{R}^m
\end{equation*}
is a rule that assigns to every $x$ in $\mathbb{R}^n$ an element $y$ in $\mathbb{R}^m$.
\end{definition}
\begin{itemize}
	\item If $y$ is the element assigned to $x$ we write $y = f(x)$ and
	\begin{equation*}
	f:x \mapsto f(x)
	\end{equation*}
	\item Two different elements $x_0$ and $x_1$ of $\mathbb{R}^n$ may be assigned the same element $y$ of $\mathbb{R}^m$. I.e:
	\begin{equation*}
	f(x_0) = y = f(x_1)
	\end{equation*}
	\item We cannot assign two different elements $y_0$ and $y_1$ of $\mathbb{R}^m$ to a single element $x$ of $\mathbb{R}^n$.
\end{itemize}
\end{frame}

\begin{frame}{Examples}
\begin{example}
\begin{equation*}
f(x) = x+1 \text{ and } f(x) = x^2+3x-5 \text{ and } f(x) = 5
\end{equation*}
are all functions $\mathbb{R}^1 \rightarrow \mathbb{R}^1$
\end{example}
\begin{example}
\begin{equation*}
f(x, y) = \left[
\begin{array}{c}
x\\
x+y\\
xy
\end{array}
\right]
\end{equation*}
is a function $f:\mathbb{R}^2 \rightarrow \mathbb{R}^3$
\end{example}
\end{frame}

\begin{frame}{Examples}
\begin{example}
If $f(x) = x^2$ then both $+1$ and $-1$ are assigned the value $+1$. I.e.
\begin{equation*}
f(-1) = 1 = f(+1)
\end{equation*}
\end{example}
\begin{example}[Non-example]
\begin{equation*}
f(x) = \begin{cases}
	-1 & \text{ if }x<1\\
	+1 & \text{ if }x>-1
\end{cases}
\end{equation*}
is not a function because the numbers between $-1$ and $+1$ are assigned two different numbers.
\end{example}
\end{frame}

\begin{frame}{Linear transformations}
\begin{definition}
A \emph{linear transformation $T:\mathbb{R}^n \rightarrow \mathbb{R}^m$} is a function satisfying:
\begin{itemize}
	\item (Linearity) $T(x+y) = T(x) + T(y)$
	\item (Scalar multiplication) $T(kx) = kT(x)$
\end{itemize}
\end{definition}
Some consequences:-
\begin{itemize}
	\item $T(0) = 0$
	\item $T(-x) = -T(x)$
	\item $T\left(\sum_{i=1}^n k_i v_i\right) = \sum_{i=1}^n k_i T(v_i)$
\end{itemize}
\end{frame}

\begin{frame}{Examples}
\begin{example}
Is the following a linear transformation?
\begin{equation*}
T \left[
\begin{array}{c}
x\\
y
\end{array}
\right] = \left[
\begin{array}{c}
2x\\
y\\
-x+2y
\end{array}
\right]
\end{equation*}
\end{example}
\begin{example}
Is the following a linear transformation?
\begin{equation*}
T \left[
\begin{array}{c}
x\\
y
\end{array}
\right] = \left[
\begin{array}{c}
xy\\
x+y
\end{array}
\right]
\end{equation*}
\end{example}
\end{frame}

\begin{frame}{Examples}
\begin{example}
Is rotation about the origin in $\mathbb{R}^2$ through an angle $\theta$ a linear transformation?
\end{example}
\begin{example}
Is reflection in the x-axis a linear transformation?
\end{example}
\begin{example}
Is $f(x) = x+1$ a linear transformation?
\end{example}
\end{frame}

\section{Matrices redux}

\begin{frame}
\begin{beamercolorbox}[sep=12pt,center]{part title}
\usebeamerfont{section title}
\insertsection\par
\end{beamercolorbox}
\end{frame}

\begin{frame}{Linear transformations determined by action on basis}
If $T$ is linear then
\begin{equation*}
T\left(\sum_{i=0}^n k_i v_i\right) = \sum_{i=0}^n k_i T(v_i)
\end{equation*}
But for all $v$ in $\mathbb{R}^n$ we can write $v$ as a linear combination of the standard basis vectors:
\begin{equation*}
v = \left[
\begin{array}{c}
v_1\\
v_2\\
v_3\\
\vdots\\
v_n
\end{array}
\right] = \sum_{i=0}^n v_i e_i = v_1 \left[
\begin{array}{c}
1\\
0\\
0\\
\vdots\\
0
\end{array}
\right] + v_2 \left[
\begin{array}{c}
0\\
1\\
0\\
\vdots\\
0
\end{array}
\right] + v_3 \left[
\begin{array}{c}
0\\
0\\
1\\
\vdots\\
0
\end{array}
\right]+\dots v_n \left[
\begin{array}{c}
0\\
0\\
0\\
\vdots\\
1
\end{array}
\right]
\end{equation*}
\end{frame}

\begin{frame}{Linear transformation determined by action on basis}
So:
\begin{equation*}
T(v) = T\left(\sum_{i=0}^n v_i e_i\right) = \sum_{i=0}^n v_i T(e_i)
\end{equation*}\vfill
{\bf Slogan:} A linear transformation is completely determined by its action on the standard basis vectors.\vfill
More explicitly: we only need to know
\begin{equation*}
T(e_1), T(e_2),\dots, T(e_n)
\end{equation*}
to work out the whole transformation!
\end{frame}

\begin{frame}{Matrices redux}
If $T:\mathbb{R}^n \rightarrow \mathbb{R}^m$ then
\begin{equation*}
T(e_1), T(e_2),\dots, T(e_n)
\end{equation*}
are in $\mathbb{R}^m$. And so the transformation is completely determined by $n\times m$ scalars\dots
\end{frame}

\begin{frame}{Matrices redux}
\begin{definition}
The \emph{matrix $[T]$ of a linear transformation $T$} is the matrix that has columns given by the images of the standard basis vectors:
\begin{equation*}
[T] = \left[
\begin{array}{cccc}
T(e_1) & T(e_2) &\dots & T(e_n)
\end{array}
\right]
\end{equation*} 
\end{definition}\vfill
(It turns out that the definition of matrix product is forced on us after this choice if we want it to match the composition of functions.)
\end{frame}

\begin{frame}{Examples}
\begin{example}
	Find
	\begin{equation*}
	T \left[
	\begin{array}{c}
	-7\\
	3\\
	-9
	\end{array}
	\right]
	\end{equation*}
	if $T$ is linear and
	\begin{equation*}
	T \left[
	\begin{array}{c}
	1\\
	3\\
	1
	\end{array}
	\right] = \left[
	\begin{array}{c}
	4\\
	4\\
	0\\
	-2
	\end{array}
	\right]\text{ and } T \left[
	\begin{array}{c}
	4\\
	0\\
	5
	\end{array}
	\right] = \left[
	\begin{array}{c}
	4\\
	5\\
	-1\\
	5
	\end{array}
	\right]
	\end{equation*}
	% Ans: [-4,-6,2,-12]
\end{example}
\end{frame}


\begin{frame}{Examples}
\begin{example}
	Find
	\begin{equation*}
	T \left[
	\begin{array}{c}
	1\\
	3\\
	-2\\
	-4
	\end{array}
	\right]
	\end{equation*}
	if $T$ is linear and
	\begin{equation*}
	T \left[
	\begin{array}{c}
	1\\
	1\\
	0\\
	-2
	\end{array}
	\right] = \left[
	\begin{array}{c}
	2\\
	3\\
	-1
	\end{array}
	\right]\text{ and } T \left[
	\begin{array}{c}
	0\\
	-1\\
	1\\
	1
	\end{array}
	\right] = \left[
	\begin{array}{c}
	5\\
	0\\
	1
	\end{array}
	\right]
	\end{equation*}
	% Ans: [-8,3,-3]
\end{example}
\end{frame}

\begin{frame}{Examples}
\begin{example}
Find the matrix of the linear transformation
\begin{equation*}
T \left[
\begin{array}{c}
x\\
y
\end{array}
\right] = \left[
\begin{array}{c}
x+2y\\
x-y
\end{array}
\right]
\end{equation*}
\end{example}
\begin{example}
Find the matrix of the linear transformation satisfying
\begin{equation*}
T \left[
\begin{array}{c}
1\\
1
\end{array}
\right] = \left[
\begin{array}{c}
1\\
2
\end{array}
\right]\text{ and } T \left[
\begin{array}{c}
0\\
-1
\end{array}
\right] = \left[
\begin{array}{c}
3\\
2
\end{array}
\right]
\end{equation*}
\end{example}
\end{frame}

\begin{frame}{Examples}
\begin{example}
For the following linear transformations $S$ and $T$ what is $S\circ T$, $T\circ S$ and what are the corresponding matrices?
\begin{equation*}
S \left[
\begin{array}{c}
x\\
y
\end{array}
\right] = \left[
\begin{array}{c}
x\\
-y
\end{array}
\right]\text{ and } T \left[
\begin{array}{c}
x\\
y
\end{array}
\right] = \left[
\begin{array}{c}
-y\\
x
\end{array}
\right]
\end{equation*}
\end{example}
\begin{example}
What is the inverse transformation and corresponding matrix for
\begin{equation*}
T \left[
\begin{array}{c}
x\\
y
\end{array}
\right] = \left[
\begin{array}{c}
x+y\\
y
\end{array}
\right]
\end{equation*}
\end{example}
\end{frame}

\begin{frame}{Examples}
\begin{example}
What is the matrix of the linear transformation for counter-clockwise rotation about the origin by an angle of $\theta$?
\end{example}
\begin{example}
What is the matrix of the linear transformation for reflection in the x-axis?
\end{example}
\begin{example}
What is the matrix of the linear transformation that is $v\times (-)$ for some vector $v\in \mathbb{R}^3$?
\end{example}
\end{frame}

\begin{frame}{Examples}
\begin{example}
What is the matrix of the linear transformation for reflection in the line $y = mx$?
\end{example}
\begin{example}
What is the matrix of the linear transformation consisting of first reflects in the x-axis and then rotates through an angle of $\frac{\pi}{2}$?
\end{example}
\begin{example}
Find the rotation or reflection that equals reflection in the line $y = -x$ followed by reflection in the y-axis.
\end{example}
\end{frame}

\begin{frame}{Summary of linear transformations}
\begin{itemize}
	\item Recall interpretation of matrix action on a vector
	\item Definition of linear transformation; examples and non-examples
	\item The matrix of a linear transformation; completely determined by action on basis
	\item Linear transformation preserves 0, - and linear combinations
	\item Composition of functions, linear transformation and relationship to matrices.
	\item Inverse functions and relationship to inverse matrices.
	\item Determinant of rotation and reflection. Composite of rotations/reflections with rotations/reflections
\end{itemize}
\end{frame}

\end{document}