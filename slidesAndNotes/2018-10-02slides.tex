\documentclass{beamer}

\useoutertheme[subsection=false]{miniframes}
\usecolortheme{beaver}
\setbeamertemplate{navigation symbols}{}
\setbeamertemplate{footline}{}
\usepackage{graphicx}
\usepackage{url}
\usepackage{datetime}

\newcommand {\framedgraphic}[3] {
  \begin{frame}{#1}
    \vspace{-0.5cm}
    \begin{center}
      \includegraphics[width=0.9\textwidth,keepaspectratio]{#2}
    \end{center}
    \vspace{-1cm}
    \begin{center}
      #3
    \end{center}
  \end{frame}
}
\newcommand{\lectureDate}{\formatdate{02}{10}{2018}}

\setbeamertemplate{caption}{\raggedright\insertcaption\par}
\title{MATH211: Linear Methods I}
\author{Matthew Burke}
\date{\lectureDate}
\begin{document}

\frame{\titlepage}

\begin{frame}{Lecture on \lectureDate}
  \tableofcontents
\end{frame}

\section*{Last time}
\label{sec:Last-time}

\begin{frame}{Last time}
  \begin{itemize}
  \item Determinants as area/volume etc..\vfill
  \item Computing determinants using row operations.\vfill
  \item The cofactor expansion (see again this time).
  \end{itemize}
\end{frame}

\section{First row expansion}

\begin{frame}
  \begin{beamercolorbox}[sep=12pt,center]{part title}
    \usebeamerfont{section title}\insertsection\par
  \end{beamercolorbox}
\end{frame}

\begin{frame}{Two dimensional determinant}
  Recall that
  \begin{equation*}
    \left|
      \begin{array}{cc}
	a&b\\
        c&d
      \end{array}
    \right| = ad-bc
  \end{equation*}
  \begin{example}
    \begin{equation*}
      \left|
	\begin{array}{cc}
          10&1\\
          9&5
	\end{array}
      \right| = 50-9 = 41
    \end{equation*}
  \end{example}
  \begin{example}
    \begin{equation*}
      \left|
	\begin{array}{cc}
          1&2\\
          3&4
	\end{array}
      \right|= 4-6 = -2
    \end{equation*}
  \end{example}
\end{frame}

\begin{frame}{Three dimensional determinants}
  \begin{definition}
    The \emph{cofactor expansion along row 1 is}
    \begin{equation*}
      \left|
	\begin{array}{ccc}
          a_{11}&a_{12}&a_{13}\\
          a_{21}&a_{22}&a_{23}\\
          a_{31}&a_{32}&a_{33}
	\end{array}
      \right| = a_{11} \left|
	\begin{array}{cc}
          a_{22}&a_{23}\\
          a_{32}&a_{33}
	\end{array}
      \right|-a_{12} \left|
	\begin{array}{cc}
          a_{21}&a_{23}\\
          a_{31}&a_{33}
	\end{array}
      \right|+a_{13} \left|
	\begin{array}{cc}
          a_{21}&a_{22}\\
          a_{31}&a_{32}
	\end{array}
      \right|
    \end{equation*}
  \end{definition}
  \begin{example}
    \begin{align*}
      \left|
	\begin{array}{ccc}
          1&1&3\\
          2&4&1\\
          5&2&6
	\end{array}
      \right| &= 1 \left|
	\begin{array}{cc}
          4&1\\
          2&6
	\end{array}
      \right| -1 \left|
	\begin{array}{cc}
          2&1\\
          5&6
	\end{array}
      \right|+3 \left|
	\begin{array}{cc}
          2&4\\
          5&2
	\end{array}
             \right|\\
           &= 1\cdot(24-2)-1\cdot(12-5)+3\cdot(4-20) \\
           &= 22-7-48 = -33
    \end{align*}
  \end{example}
\end{frame}

\begin{frame}{Examples}
  \begin{example}
    Find
    \begin{equation*}
      \left|
	\begin{array}{ccc}
          1&2&3\\
          0&5&6\\
          0&0&9
	\end{array}
      \right|
    \end{equation*}
  \end{example}\vfill
  \begin{example}
    Find
    \begin{equation*}
      \left|
	\begin{array}{ccc}
          1&2&3\\
          4&5&6\\
          7&8&9
	\end{array}
      \right|
    \end{equation*}
  \end{example}
\end{frame}

\begin{frame}{Expansions in general}
  \begin{definition}[Non-standard]
    If $A$ is a matrix write $A(i|j)$ for the matrix obtained from $A$ by deleting the $i$th row and the $j$th column.
  \end{definition}
  \begin{definition}
    The \emph{cofactor expansion of $A$ along the first row is}
    \begin{align*}
      \left|A\right| = \sum_{k=1}^na_{1k}(-1)^{1+k}\left| A(1| k) \right|
    \end{align*}
    which when expanded is:
    \begin{equation*}
      \left|A\right|= a_{11}\left| A(1| 1) \right|-a_{12}\left| A(1| 2) \right|+a_{13}\left| A(1| 3) \right|-\dots+(-1)^{1+n}a_{1n}\left| A(1| n) \right|
    \end{equation*}
  \end{definition}
\end{frame}

\begin{frame}{Auxiliary definitions}
  \begin{definition}[Minor]
    \begin{equation*}
      minor_{ij}(A) = \left|A(i| j)\right|
    \end{equation*}
  \end{definition}\vfill
  \begin{definition}[Cofactor]
    \begin{align*}
      cof_{ij}(A) &= (-1)^{i+j}minor_{ij}(A)\\
                  &= (-1)^{i+j}\left|A(i| j)\right|
    \end{align*}
  \end{definition}\vfill
  \begin{lemma}[Expansion along row 1]
    \begin{equation*}
      \left|A\right| = \sum_{k=1}^na_{1k}cof_{1k}(A)
    \end{equation*}
  \end{lemma}
\end{frame}

\begin{frame}{Examples}
  \begin{example}
    Find
    \begin{equation*}
      \left|
	\begin{array}{cccc}
          0&1&-2&1\\
          5&0&0&7\\
          0&1&-1&0\\
          3&0&0&2
	\end{array}
      \right|
    \end{equation*}
  \end{example}
  \begin{example}
    Find
    \begin{equation*}
      \left|
	\begin{array}{cccc}
          2&5&4&3\\
          0&4&1&7\\
          0&0&3&2\\
          0&0&0&5
	\end{array}
      \right|
    \end{equation*}
  \end{example}
\end{frame}

\begin{frame}
  Questions?
\end{frame}

\section{Arbitrary expansion}

\begin{frame}
  \begin{beamercolorbox}[sep=12pt,center]{part title}
    \usebeamerfont{section title}\insertsection\par
  \end{beamercolorbox}
\end{frame}

\begin{frame}{General cofactor expansion}
  In fact we may use the cofactors of arbitrary rows and columns:\vfill
  \begin{lemma}[Expansion along arbitrary row or column]
    \begin{align*}
      \left|A\right| &= \sum_{k=1}^n a_{ik} (-1)^{i+k}\left|A(i| k)\right|\\
                     &= \sum_{k=1}^n a_{kj}(-1)^{k+j}\left|A(k| j)\right|
    \end{align*}
  \end{lemma}\vfill
  So we can choose the row or column which is most convenient.
\end{frame}


\begin{frame}{Examples}
  \begin{example}
    Find
    \begin{equation*}
      \left|
	\begin{array}{ccc}
          1&2&3\\
          0&5&6\\
          0&0&9
	\end{array}
      \right|
    \end{equation*}
    by expanding along row 3 or column 1.
  \end{example}\vfill
  \begin{example}
    Find
    \begin{equation*}
      \left|
	\begin{array}{cccc}
          0&1&-2&1\\
          5&0&0&7\\
          0&1&-1&0\\
          3&0&0&2
	\end{array}
      \right|
    \end{equation*}
    by expanding along column 2.
  \end{example}
\end{frame}

\begin{frame}{Example}
  \begin{example}
    Find
    \begin{equation*}
      \left|
	\begin{array}{cccc}
          -8&1&0&-4\\
          5&7&0&-7\\
          12&-3&0&8\\
          -3&11&0&2
	\end{array}
      \right|
    \end{equation*}
    by choosing the correct column\dots
  \end{example}
\end{frame}

\begin{frame}
  Questions?
\end{frame}

\section{Inversion formula}

\begin{frame}
  \begin{beamercolorbox}[sep=12pt,center]{part title}
    \usebeamerfont{section title}\insertsection\par
  \end{beamercolorbox}
\end{frame}

\begin{frame}{Duplicating rows}
  Suppose that an $n\times n$ matrix has $i$th row equal to $j$th row:
  \begin{equation*}
    A = \left[
      \begin{array}{cccc}
        a_{11} & a_{12} & \dots & a_{1n}\\
        a_{21} & a_{22} & \dots & a_{2n}\\
        \dots & \dots & \dots & \dots\\
        {\color{blue}a_{i1}} & {\color{blue}a_{i2}} & {\color{blue}\dots} & {\color{blue}a_{in}}\\
        \dots & \dots & \dots & \dots\\
        {\color{blue}a_{i1}} & {\color{blue}a_{i2}} & {\color{blue}\dots}& {\color{blue}a_{in}}\\
        \dots & \dots & \dots & \dots\\
        a_{n1} & a_{n2} & \dots & a_{nn}\\
      \end{array}
    \right]
  \end{equation*}
  then we know that $\left|A\right| = 0$. (Why?)
\end{frame}

\begin{frame}{Adjugate matrix}
  If we expand
  \begin{equation*}
    A = \left[
      \begin{array}{cccc}
        a_{11} & a_{12} & \dots & a_{1n}\\
        a_{21} & a_{22} & \dots & a_{2n}\\
        \dots & \dots & \dots & \dots\\
        {\color{blue}a_{i1}} & {\color{blue}a_{i2}} & {\color{blue}\dots} & {\color{blue}a_{in}}\\
        \dots & \dots & \dots & \dots\\
        {\color{blue}a_{i1}} & {\color{blue}a_{i2}} & {\color{blue}\dots}& {\color{blue}a_{in}}\\
        \dots & \dots & \dots & \dots\\
        a_{n1} & a_{n2} & \dots & a_{nn}\\
      \end{array}
    \right]
  \end{equation*}
  along the $j$th row then we get:
  \begin{align*}
    0 = \left|A\right| &= \sum_{k=1}^n a_{jk}(-1)^{j+k}\left|A(j|k)\right|\\
                   &= \sum_{k=1}^n a_{ik}(-1)^{j+k}\left|A(j|k)\right|
  \end{align*}
\end{frame}

\begin{frame}{The adjugate matrix}
  Now the last term looks like a matrix product:
  \begin{equation*}
    \sum_{k=1}^n a_{ik}(-1)^{j+k}\left|A(j|k)\right| = (A(adj (A)))_{ij}
  \end{equation*}
  \begin{definition}[Adjugate matrix]
    \begin{align*}
      adj (A)_{ij} &= (-1)^{i+j}\left|A(j|i)\right|\\
                   &= cof_{ji}(A)
    \end{align*}
  \end{definition}
  So $adj(A)$ is the transpose of the matrix of cofactors $cof_{ij}(A)$.
\end{frame}

\begin{frame}{A theorem}
  \begin{theorem}
    \begin{equation*}
      A\cdot adj(A) = adj(A)\cdot A = det(A) I
    \end{equation*}
    \begin{proof}
      Just compute the ij-entries:
      \begin{align*}
        (A\cdot adj(A))_{ij} &= \sum_{k=1}^n a_{ik}adj(A)_{kj}\\
                             &= \sum_{k=1}^n a_{ik}(-1)^{j+k}\left|A(j|k)\right|
      \end{align*}
    \end{proof}
  \end{theorem}
\end{frame}

\begin{frame}{A theorem continued...}
  \begin{theorem}
    \begin{equation*}
      A\cdot adj(A) = adj(A)\cdot A = det(A) I
    \end{equation*}
    \begin{proof}
      ...so on the diagonal (when $i=j$) the entry is
      \begin{equation*}
        \sum_{k=1}^n a_{ik}(-1)^{i+k}\left|A(i|k)\right| = det(A)
      \end{equation*}
      and off the diagonal
      \begin{equation*}
        \sum_{k=1}^n a_{ik}(-1)^{j+k}\left|A(j|k)\right| = 0
      \end{equation*}
      as required.
    \end{proof}
  \end{theorem}
\end{frame}

\begin{frame}{The inversion formula}
  So we know that
  \begin{equation*}
    A\cdot adj(A) = adj(A)\cdot A = det(A) I
  \end{equation*}\vfill
  and therefore if $det(A)\neq 0$ then
  \begin{equation*}
    A^{-1} = \frac{1}{det(A)} adj(A)
  \end{equation*}\vfill
  {\bf Warning:} for $n>2$ this is a very inefficient way to compute an inverse.
\end{frame}

\begin{frame}{Examples}
  \begin{example}
    Compute the inverse of the following matrices:-
    \begin{equation*}
      \left[
	\begin{array}{ccc}
          4&0&3\\
          1&9&7\\
          0&6&4
	\end{array}
      \right]
    \end{equation*}
    \begin{equation*}
      \left[
	\begin{array}{cc}
          a&b\\
          c&d
	\end{array}
      \right]
    \end{equation*}
    \begin{equation*}
      \left[
	\begin{array}{ccc}
          2&1&3\\
          5&-7&1\\
          3&0&-6
	\end{array}
      \right]
    \end{equation*}
  \end{example}
\end{frame}

\begin{frame}
  Questions?
\end{frame}

\end{document}