\documentclass{beamer}

\useoutertheme[subsection=false]{miniframes}
\usecolortheme{beaver}
\setbeamertemplate{navigation symbols}{}
\setbeamertemplate{footline}{}
\usepackage{graphicx}
\usepackage{url}
\usepackage{datetime}
\usepackage{tikz-cd}
\newcommand{\lectureDate}{\formatdate{18}{10}{2018}}

\setbeamertemplate{caption}
{\raggedright\insertcaption\par}
\title{MATH211: Linear Methods I}
\author{Matthew Burke}
\date{\lectureDate}
\begin{document}

\frame{\titlepage}

\begin{frame}{Lecture on \lectureDate}
  \tableofcontents
\end{frame}

\section*{Last time}
\label{sec:Last-time}

\begin{frame}{Last time}
  \begin{itemize}
  \item Orthogonality\vfill
  \item Projections\vfill
  \item Closest points
  \end{itemize}
\end{frame}

\section{Closest points}

\begin{frame}
\begin{beamercolorbox}[sep=12pt,center]{part title}
\usebeamerfont{section title}
\insertsection\par
\end{beamercolorbox}
\end{frame}

\begin{frame}{Line given by parametric equations}
    \begin{columns}
        \hspace{-1cm}
        \column{0.5\textwidth}
        \includegraphics[scale=0.6]{2d-parametric-closest.png}
        \column{0.4\textwidth}
        First we project onto the direction of the line:
        \begin{equation*}
            \overrightarrow {PA} = \frac{d}{\|d\|}\cdot (Q-P)\frac{d}{\|d\|}
        \end{equation*}
        and then we can find $A$:
        \begin{equation*}
            A = P + \frac{d}{\|d\|}\cdot (Q-P)\frac{d}{\|d\|}
        \end{equation*}
        (Use same technique for line in $\mathbb R^n$ also.)
    \end{columns}
\end{frame}

\begin{frame}{Line given by normal}
\begin{columns}
    \hspace{-1cm}
    \column{0.5\textwidth}
    \includegraphics[scale=0.6]{2d-normal-closest.png}
    \column{0.4\textwidth}
    In this picture
    \[
        \overrightarrow{PB}\|\overrightarrow{AQ}\text{ and } \overrightarrow{PA}\| \overrightarrow{BQ}
    \]  
    First we project onto the normal:
    \begin{equation*}
        \overrightarrow {PB} = \frac{n}{\|n\|}\cdot (Q-P)\frac{n}{\|n\|}
    \end{equation*}
    and then we can find $A$:
    \begin{equation*}
        A = Q - \frac{n}{\|n\|}\cdot (Q-P)\frac{n}{\|n\|}
    \end{equation*}
    (Use same technique for $(n-1)$-dimensional hyperplane in $\mathbb R^n$ also.)
\end{columns}
\end{frame}

\begin{frame}{Closest points on skew lines}
\begin{columns}
    \hspace{-1cm}
    \column{0.5\textwidth}
    \includegraphics[scale=0.3]{skwe-lines.png}
    \column{0.4\textwidth}
    Key observation: if $A$ and $B$ are the closest points then
    \begin{equation*}
    \overrightarrow {AB} \perp d\text{ and } \overrightarrow {AB} \perp e
    \end{equation*}
\end{columns}
\end{frame}

\begin{frame}{Closest points on skew lines}
\begin{columns}
    \hspace{-1cm}
    \column{0.5\textwidth}
    \includegraphics[scale=0.3]{skwe-lines.png}
    \column{0.4\textwidth}
    \hspace{1cm}
    Firstly:
    \begin{equation*}
    A = P+sd\text{ and } B = Q+te
    \end{equation*}
    and secondly
    \begin{align*}
    (B-A)\cdot d = 0\\
    (B-A)\cdot e = 0
    \end{align*}
    so we get two equations in two unknowns $s$ and $t$.
\end{columns}
\end{frame}

\begin{frame}{Example}
    \begin{example}
    Find the closest point to
    \begin{equation*}
      Q=\left[
	\begin{array}{c}
          3\\
          2\\
          -1
	\end{array}
      \right]
    \end{equation*}
    on the line
    \begin{equation*}
      \left[
	\begin{array}{c}
          x_1\\
          x_2\\
          x_3
	\end{array}
      \right] = \left[
	\begin{array}{c}
          2\\
          1\\
          3
	\end{array}
      \right]+s \left[
	\begin{array}{c}
          3\\
          -1\\
          -2
	\end{array}
      \right]
    \end{equation*}
  \end{example}
\end{frame}

\begin{frame}{Example}
    \begin{example}
        Find the closest point to
        \begin{equation*}
          \left[
        \begin{array}{c}
              2\\
              3
        \end{array}
          \right]
        \end{equation*}
        on the line with equation $4x+3y = -3$.
      \end{example}
    \begin{example}
    Find the closest point to
    \begin{equation*}
      \left[
	\begin{array}{c}
          2\\
          3\\
          0
	\end{array}
      \right]
    \end{equation*}
    on the plane with equation $5x+y+z = -1$.
  \end{example}
\end{frame}

\begin{frame}{Examples}
\begin{example}
Given the two lines
\begin{equation*}
\left[
\begin{array}{c}
x\\
y\\
z
\end{array}
\right] = \left[
\begin{array}{c}
3\\
1\\
-1
\end{array}
\right]+s \left[
\begin{array}{c}
1\\
1\\
-1
\end{array}
\right]\text{ and } \left[
\begin{array}{c}
x\\
y\\
z
\end{array}
\right] = \left[
\begin{array}{c}
1\\
2\\
0
\end{array}
\right] +t \left[
\begin{array}{c}
1\\
0\\
2
\end{array}
\right]
\end{equation*}
find points $A$ on the first line and $B$ on the second line such that the distance $A$ to $B$ is minimised.
\end{example}
\end{frame}

\section{Cross product}

\begin{frame}
\begin{beamercolorbox}[sep=12pt,center]{part title}
\usebeamerfont{section title}
\insertsection\par
\end{beamercolorbox}
\end{frame}

\begin{frame}{Motivation for cross product}
The \emph{cross product} $u\times v$ measures how non-parallel $u$ and $v$ are.
\begin{equation*}
u \| v \iff \text{there exists a scalar $k$ such that } ku = v
\end{equation*}
which in $\mathbb{R}^3$ means
\begin{align*}
k u_1 = v_1 && &&u_1v_2 - u_2v_1 = 0\\
ku_2 = v_2 &&\implies & &u_2v_3 - u_3v_2 = 0\\
ku_3 = v_3 && & &u_3v_1 - u_1v_3 = 0
\end{align*}
\end{frame}

\begin{frame}{Definition of cross product}
\begin{definition}
The \emph{cross product $u\times v$ of $u$ with $v$} is
\begin{equation*}
\left[
\begin{array}{c}
u_1\\
u_2\\
u_3
\end{array}
\right]\times \left[
\begin{array}{c}
v_1\\
v_2\\
v_3
\end{array}
\right] = \left[
\begin{array}{c}
u_2v_3-u_3v_2\\
u_3v_1-v_3u_1\\
u_1v_2-u_2v_1
\end{array}
\right]
\end{equation*}
\textbf{Note that the the order has changed.}
\end{definition}
\end{frame}

\begin{frame}{Trigonometric definition of cross product}
\begin{columns}
  \column{0.5\textwidth}
  \includegraphics{cross-product-direction.png}
  \column{0.4\textwidth}
  Alternatively:
  \begin{equation*}
  u\times v = \|u\|\|v\| sin\theta n
  \end{equation*}
  where $n$ is the unit vector given by the `right hand rule'.
\end{columns}
\end{frame}

\begin{frame}{Properties of the cross product}
\begin{theorem}
Let $\vec{u}, \vec{v}$ and $\vec{w}$ be in $\mathbb{R}^3$.
\begin{enumerate}
\item $\vec{u}\times\vec{v}$ is a vector.
\item $\vec{u}\times\vec{v}$ is orthogonal to both $\vec{u}$ and $\vec{v}$.
\item $\vec{u}\times\vec{0}=\vec{0}$ and
$\vec{0}\times\vec{u}=\vec{0}$.
\item $\vec{u}\times\vec{u}=\vec{0}$.
\item $\vec{u}\times\vec{v} = - (\vec{v}\times\vec{u})$.
\item $(k\vec{u})\times\vec{v}
= k(\vec{u}\times\vec{v})
=\vec{u}\times(k\vec{v})$ for any scalar $k$.
\item $\vec{u}\times(\vec{v} + \vec{w}) =
\vec{u}\times\vec{v} + \vec{u}\times\vec{w}$.
\item $(\vec{v} + \vec{w})\times\vec{u}=
\vec{v}\times\vec{u} + \vec{w}\times\vec{u}$.
\end{enumerate}
\end{theorem}
\end{frame}

\begin{frame}{Example}
\begin{example}
  Find $u\times v$ where
  \begin{equation*}
  u= \left[
  \begin{array}{c}
  1\\
  -1\\
  2
  \end{array}
  \right]\text{ and } v = \left[
  \begin{array}{c}
  3\\
  -2\\
  1
  \end{array}
  \right]
  \end{equation*}
\end{example}
\begin{example}
Find all vectors orthogonal to 
\begin{equation*}
\left[
\begin{array}{c}
1\\
-3\\
2
\end{array}
\right]\text{ and } \left[
\begin{array}{c}
0\\
1\\
1
\end{array}
\right]
\end{equation*}
\end{example}
\end{frame}

\begin{frame}{Examples}
\begin{example}
  Use a normal vector to give an equation for the plane passing through
  \begin{equation*}
  \left[
  \begin{array}{c}
  1\\
  -1\\
  2
  \end{array}
  \right]\text{ , } \left[
  \begin{array}{c}
  2\\
  0\\
  -1
  \end{array}
  \right]\text{ and }
  \left[
  \begin{array}{c}
  0\\
  -2\\
  3
  \end{array}
  \right]
  \end{equation*}
\end{example}
\end{frame}

\section{Applications}

\begin{frame}
\begin{beamercolorbox}[sep=12pt,center]{part title}
\usebeamerfont{section title}
\insertsection\par
\end{beamercolorbox}
\end{frame}

\begin{frame}{Area of a parallelogram}
\begin{columns}
  \column{0.5\textwidth}
  \includegraphics{parallelogram.png}
  \column{0.4\textwidth}
  The area of the parallelogram is:
  \begin{equation*}
  \|u\|\|v\|sin\theta = \|u\times v\|
  \end{equation*}
\end{columns}
\end{frame}

\begin{frame}{Triple scalar (box) product}
\begin{definition}
If $u$, $v$ and $w$ are vectors in $\mathbb{R}^3$ then the \emph{triple scalar product (or box product) of $u$, $v$ and $w$} is
\begin{equation*}
u\cdot (v\times w)
\end{equation*}
\end{definition}
Note that the order of the vectors matters. In fact
\begin{lemma}
\begin{equation*}
u\cdot (v\times w) = \left|
\begin{array}{ccc}
u_1 &v_1 &w_1\\
u_2 &v_2 &w_2\\
u_3&v_3&w_3
\end{array}
\right|
\end{equation*}
which shows that it is the signed volume of the parallelepiped generated by $u$, $v$ and $w$.
\end{lemma}
\end{frame}

\begin{frame}{Examples}
\begin{example}
Find the area of the triangle having vertices
\begin{equation*}
\left[
\begin{array}{c}
3\\
-1\\
2
\end{array}
\right], \left[
\begin{array}{c}
1\\
1\\
0
\end{array}
\right]\text{ and } \left[
\begin{array}{c}
1\\
2\\
-1
\end{array}
\right]
\end{equation*}
\end{example}
\begin{example}
Find the volume of the parallelepiped determined by the vectors
\begin{equation*}
\left[
\begin{array}{c}
2\\
1\\
-1
\end{array}
\right], \left[
\begin{array}{c}
1\\
0\\
2
\end{array}
\right]\text{ and } \left[
\begin{array}{c}
2\\
1\\
1
\end{array}
\right]
\end{equation*}
\end{example}
\end{frame}

\section{Misc}

\begin{frame}
\begin{beamercolorbox}[sep=12pt,center]{part title}
\usebeamerfont{section title}
\insertsection\par
\end{beamercolorbox}
\end{frame}

\begin{frame}{Cauchy-Schwarz inequality}
\begin{columns}
    \hspace{-1cm}
    \column{0.5\textwidth}
    \includegraphics[scale=1.7]{projection-onto-unit.png}
    \column{0.45\textwidth}
    If $e$ is a unit vector then
    \begin{equation*}
    |e\cdot v| \leq \|v\|
    \end{equation*}
    More generally:
    \begin{equation*}
        |u\cdot v| \leq \|u\|\|v\|
    \end{equation*}
    which is the \emph{Cauchy-Schwarz inequality}.
\end{columns}
\end{frame}

\begin{frame}{Triangle inequality}
\begin{theorem}
  If $u$ and $v$ are vectors in $\mathbb{R}^n$ then
  \begin{equation*}
    \|u+v\| \leq \|u\|+\|v\|
  \end{equation*}
\end{theorem}\vfill
\begin{theorem}
  If $u$ and $v$ are vectors in $\mathbb{R}^n$ then
  \begin{equation*}
    |\|u\|-\|v\||\leq \|u-v\|
  \end{equation*}
\end{theorem}
\end{frame}

\begin{frame}{The Lagrange identity}
\begin{theorem}
If $u$ and $v$ are in $\mathbb{R}^3$ then
\begin{equation*}
\|u\times v\|^2 = \|u\|^2\|v\|^2-(u\cdot v)^2
\end{equation*}
\end{theorem}
\end{frame}

\begin{frame}{Example}
    \begin{example}
        Find equations for the lines through
        \begin{equation*}
            \left[
            \begin{array}{c}
            1\\
            0\\
            1
            \end{array}
            \right]
        \end{equation*}
        that meet the line
        \begin{equation*}
            \left[
            \begin{array}{c}
            x\\
            y\\
            z
            \end{array}
            \right] = \left[
            \begin{array}{c}
            1\\
            2\\
            0
            \end{array}
            \right]+s \left[
            \begin{array}{c}
            2\\
            -1\\
            2
            \end{array}
            \right]
        \end{equation*}
        at the two points distance three from $(1, 2, 0)$.
    \end{example}
\end{frame}

\begin{frame}{Examples}
    \begin{example}
        The diagonals of a parallelogram bisect each other.
      \end{example}
      \begin{example}
        If $ABCD$ is an arbitrary quadrilateral then the midpoints of the four sides form a parallelogram.
      \end{example}
\end{frame}



\end{document}