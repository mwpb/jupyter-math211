\documentclass{beamer}

\useoutertheme[subsection=false]{miniframes}
\usecolortheme{beaver}
\setbeamertemplate{navigation symbols}{}
\setbeamertemplate{footline}{}
\usepackage{graphicx}
\usepackage{url}
\usepackage{datetime}

\newcommand {\framedgraphic}[3] {
  \begin{frame}{#1}
    \vspace{-0.5cm}
    \begin{center}
      \includegraphics[width=0.9\textwidth,keepaspectratio]{#2}
    \end{center}
    \vspace{-1cm}
    \begin{center}
      #3
    \end{center}
  \end{frame}
}
\newcommand{\lectureDate}{\formatdate{18}{09}{2018}}

\setbeamertemplate{caption}{\raggedright\insertcaption\par}
\title{MATH211: Linear Methods I}
\author{Matthew Burke}
\date{\lectureDate}
\begin{document}

\frame{\titlepage}

\begin{frame}{Lecture on \lectureDate}
  \tableofcontents
\end{frame}

\section*{Last time}
\label{sec:Last-time}
\begin{frame}{Last time}
  \begin{itemize}
  \item Matrix addition and scalar multiplication\vfill
  \item Relationship to systems of equations\vfill
  \item Matrix composition (product)\vfill
  \item Non-commutativity of matrix composition\vfill
  \item (Two sets of notes for 11th September)
  \end{itemize}
\end{frame}

\section{Non-commutativity}

\begin{frame}
  \begin{beamercolorbox}[sep=12pt,center]{part title}
    \usebeamerfont{section title}\insertsection\par
  \end{beamercolorbox}
\end{frame}

\begin{frame}{Non-commutativity}
  \begin{enumerate}
  \item $AB$ exists but $BA$ does not.\vfill
  \item Both $AB$ and $BA$ exist but are of different sizes.\vfill
  \item Both $AB$ and $BA$ exist and are the same size, but they are different.\vfill
  \item $AB = BA$ (it does happen sometimes!)
  \end{enumerate}
\end{frame}

\section{(i,j)-entry proofs}

\begin{frame}
  \begin{beamercolorbox}[sep=12pt,center]{part title}
    \usebeamerfont{section title}\insertsection\par
  \end{beamercolorbox}
\end{frame}

\begin{frame}{Entries of product matrix}
  \begin{definition}
    If $A$ and $B$ are matrices such that
    \begin{equation*}
      \text{No. of rows}(A) = m = \text{No. of columns}(B)
    \end{equation*}
    then
    \begin{align*}
      (BA)_{ij} &= \sum_{k=1}^m b_{ik}a_{kj}\\
                &= b_{i1}a_{1j} + b_{i2}a_{2j} + b_{i3}a_{3j} + \dots + b_{im}a_{mj}
    \end{align*}
  \end{definition}
\end{frame}

\begin{frame}{Picture}
  \begin{equation*}
      (BA)_{ij} = \sum_{k=1}^m {\color{red}b_{ik}}{\color{blue}a_{kj}} = {\color{red}b_{i1}}{\color{blue}a_{1j}} + {\color{red}b_{i2}}{\color{blue}a_{2j}} + {\color{red}b_{i3}}{\color{blue}a_{3j}} + \dots + {\color{red}b_{im}}{\color{blue}a_{mj}}
    \end{equation*}\vfill
    is the dot product of the $i$th row of $B$ with the $j$th column of $A$:\vfill
    \begin{equation*}
      \left[
        \begin{array}{cccc}
          b_{11} & b_{12} & \dots & b_{1m}\\
          b_{21} & b_{22} & \dots & b_{2m}\\
          \dots & \dots & \dots & \dots\\
          {\color{red}b_{i1}} & {\color{red}b_{i2}} & {\color{red}\dots} & {\color{red}b_{im}}\\
          \dots & \dots & \dots & \dots\\
          b_{p1} & b_{p2} & \dots & b_{pm}
        \end{array}
      \right]
      \left[
        \begin{array}{cccccc}
          a_{11} & a_{12} & \dots & {\color{blue}a_{1j}} & \dots & a_{1m}\\
          a_{21} & a_{22} & \dots & {\color{blue}a_{2j}} & \dots & a_{2m}\\
          \dots & \dots & \dots & {\color{blue}\dots}& \dots & \dots\\
          a_{n1} & a_{n2} & \dots & {\color{blue}a_{nj}} & \dots & a_{nm}
        \end{array}
      \right]
    \end{equation*}
\end{frame}

\begin{frame}{Examples of entries of matrix product}
    \begin{example}
      Find the $(2,3)$-entry of 
      \[ 
        \left[
          \begin{array}{rrr}
            -1 & 0 & 3 \\
            {2} & {-1} & {1}
          \end{array}
        \right]
        \left[
          \begin{array}{rrr}
            -1 & 1 & {2} \\
            0 & -2 & {4} \\
            1 & 0 & {0}
          \end{array}\right] 
      \]
    \end{example}
        \begin{example}
      Find the $(3,1)$-entry of 
      \[ 
        \left[
          \begin{array}{rrr}
            -1 & 0 & 3 \\
            0 & 0 & {1}\\
            5 & 4 & 9
          \end{array}
        \right]
        \left[
          \begin{array}{rrr}
            12 & 1 & {2} \\
            4 & 0 & {4} \\
            -2 & 0 & {0}
          \end{array}\right] 
      \]
  \end{example}
\end{frame}

\begin{frame}{General examples of (i,j)-entry}
  In the following examples assume that the matrices are of the appropriate sizes so that the products involved make sense.\vfill
  \begin{example}
    Find the $(i,j)$-entry of
    \begin{equation*}
      (CB)A
    \end{equation*}
  \end{example}\vfill
  \begin{example}
    Find the $(i,j)$-entry of
    \begin{equation*}
      C(A+B)
    \end{equation*}
  \end{example}
\end{frame}

\begin{frame}
  Questions?
\end{frame}

\begin{frame}{Transpose of a matrix}
  \begin{definition}
    If $A$ is \emph{any} matrix the \emph{transpose $A^T$ of $A$} has entries
    \begin{equation*}
      (A^T)_{ij} = A_{ji}
    \end{equation*}
  \end{definition}
  \begin{example}
    \begin{equation*}
      \left[
        \begin{array}{ccc}
          5 & 2 & -3\\
          -1 & 0 & 2
        \end{array}
      \right] =
      \left[
        \begin{array}{cc}
          5 & -1\\
          2 & 0\\
          -3 & 2
        \end{array}
      \right]
    \end{equation*}
  \end{example}
  In short: the columns of $A^T$ are the rows of $A$.
\end{frame}

\begin{frame}{Symmetric matrices}
  \begin{definition}
    A matrix $A$ is \emph{symmetric} iff $A = A^T$.\\
    In terms of entries: $A_{ij} = A_{ji}$.
  \end{definition}
  \begin{example}
    \begin{equation*}
      \left[
      \begin{array}{ccc}
        1 & 4 & -2\\
        4 & -3 & 17\\
        -2 & 17 & 0
      \end{array}
      \right]
    \end{equation*}
    is symmetric but
    \begin{equation*}
      \left[
      \begin{array}{ccc}
	1 & 4 & -2\\
        4 & -3 & 5\\
        -2 & 17 & 0
      \end{array}
      \right]
    \end{equation*}
    is not.
  \end{example}
\end{frame}

\begin{frame} {Examples of (i,j)-entry proofs}
  \begin{example}
    Prove $C(B+A) = CB+CA$.
  \end{example}
  \begin{example}
    Prove $(BA)^T = A^TB^T$.
  \end{example}
  \begin{example}
    Prove $(B+A)^T = B^T + A^T$.
  \end{example}
  \begin{example}
    Prove that if $A$ and $B$ are symmetric then $A^T+2B$ is symmetric.
  \end{example}
  \begin{example}
    Prove $C(BA) = (CB)A$. (Called matrix associativity.)
  \end{example}
\end{frame}

\section{Identity and inverse matrices}

\begin{frame}
  \begin{beamercolorbox}[sep=12pt,center]{part title}
    \usebeamerfont{section title}\insertsection\par
  \end{beamercolorbox}
\end{frame}

\begin{frame}{Identity matrix}
  \begin{definition}
    The \emph{$n\times n$ identity matrix} is:
    \begin{equation*}
      I_n = 
      \left[
	\begin{array}{cccc}
          1 & 0 & \dots & 0\\
          0 & 1 & \dots & 0\\
          \dots & \dots & \dots & \dots\\
          0 & 0 & \dots & 1
	\end{array}
      \right]
    \end{equation*}
    In terms of entries:
    \begin{equation*}
      (I_n)_{ij} = \begin{cases}
        1 & \text{if }i=j\\
        0 & \text{if }i\neq j
      \end{cases}
    \end{equation*}
  \end{definition}
\end{frame}

\begin{frame}{Product with identity}
  Whenever the following products make sense:
  \begin{align*}
    BI_m = B & & \text{ and } & & I_mA = A
  \end{align*}
  Indeed for the first:
  \begin{align*}
    (BI_m)_{ij} &= \sum_{k=1}^mB_{ik}I_{kj}\\
                &= \sum_{k\neq j}B_{jk}I_{kj} + B_{ij}I_{jj}\\
                &= \sum_{k\neq j}0 + B_{ij} = B_{ij}
  \end{align*}
\end{frame}

\begin{frame}{Inverse matrix}
  \begin{definition}
    If $A$ is an $n\times n$ matrix then an \emph{inverse matrix $A^{-1}$ of $A$} is any matrix such that
    \begin{align*}
     AA^{-1} = I_n & & \text{ and } & & A^{-1}A = I_n
    \end{align*}
  \end{definition}\vfill
  {\color{red}Some remarks:}
  \begin{itemize}
  \item It doesn't make sense to ask for an inverse of a non-square matrix.
    (Think in terms of linear transformations.)
  \item Not all square matrices have inverses.
  \item If $A^{-1}$ is the inverse of $A$ then $A$ is the inverse of $A^{-1}$.
  \end{itemize}
\end{frame}

\begin{frame}{Examples of non-existence}
  \begin{example}
    No matrix that only has zero entries has an inverse.
  \end{example}
  \begin{example}
    The following matrix does not have an inverse
    \begin{equation*}
      \left[
	\begin{array}{cc}
          0 & 1 \\
          0 & 0
	\end{array}
      \right]
    \end{equation*}
  \end{example}
  \bf{In fact a $n\times n$ matrix has an inverse iff $rank(A)=n$.}
\end{frame}

\begin{frame}{Uniqueness of inverse}
  \begin{lemma}
    If a matrix $A$ has an inverse then it is unique.
  \end{lemma}
\end{frame}

\begin{frame}
  Questions?
\end{frame}

\section{Inversion examples}
\label{sec:Inversion-algorithm}

\begin{frame}{Motivation from linear combinations}
  Lets try to find an inverse for
  \begin{equation*}
    \left[
      \begin{array}{cc}
        1&2\\
        3&4
      \end{array}
    \right]
  \end{equation*}
  We want a matrix $A$ such that
  \begin{align*}
    \left[
      \begin{array}{cc}
        1&2\\
        3&4
      \end{array}
    \right]A& =
    \left[
      \hspace{0.5cm}
      \left[
      \begin{array}{cc}
        1&2\\
        3&4
      \end{array}
      \right]
      \left[
        \begin{array}{c}
          a_{11}\\
          a_{21}
        \end{array}
      \right]
      \hspace{0.5cm}
      \left[
        \begin{array}{cc}
          1&2\\
          3&4
        \end{array}
      \right]
      \left[
        \begin{array}{c}
          a_{12}\\
          a_{22}
        \end{array}
      \right]
      \hspace{0.5cm}
    \right]\\
    &= \left[
      \begin{array}{cc}
        1&0\\
        0&1
      \end{array}
    \right]
  \end{align*}
\bf{Express the standard basis vectors in terms of the columns.}
\end{frame}

\begin{frame}{Examples}
  \begin{example}
    Find the inverse for
    \begin{equation*}
      \left[
        \begin{array}{cc}
          1&2\\
          3&4
        \end{array}
      \right]
    \end{equation*}
  \end{example}
  \begin{example}
    Find the inverse for the general 2x2 matrix
    \begin{equation*}
      \left[
        \begin{array}{cc}
          a&b\\
          c&d
        \end{array}
      \right]
    \end{equation*}
  \end{example}
\end{frame}

\begin{frame}{Examples}
  \begin{example}
    Find the inverse for
    \begin{equation*}
      \left[
        \begin{array}{cc}
          2&4\\
          1&1
        \end{array}
      \right]
    \end{equation*}
  \end{example}
  \begin{example}
    Find the inverse for
    \begin{equation*}
      \left[
        \begin{array}{cc}
          5&4\\
          10&8
        \end{array}
      \right]
    \end{equation*}
  \end{example}
  \begin{example}
    Find the inverse for
    \begin{equation*}
      \left[
        \begin{array}{ccc}
          1&0&3\\
          2&3&4\\
          1&0&2
        \end{array}
      \right]
    \end{equation*}
  \end{example}

\end{frame}

\end{document}