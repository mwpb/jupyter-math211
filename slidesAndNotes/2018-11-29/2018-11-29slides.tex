\documentclass{beamer}

\useoutertheme[subsection=false]{miniframes}
\usecolortheme{beaver}
\setbeamertemplate{navigation symbols}{}
\setbeamertemplate{footline}{}
\usepackage{graphicx}
\usepackage{url}
\usepackage{datetime}
\usepackage{tikz-cd}
\newcommand{\lectureDate}{\formatdate{29}{11}{2018}}

\setbeamertemplate{caption}
{\raggedright\insertcaption\par}
\title{MATH211: Linear Methods I}
\author{Matthew Burke}
\date{\lectureDate}
\begin{document}

\frame{\titlepage}

\begin{frame}{Lecture on \lectureDate}
  \tableofcontents
\end{frame}

\section*{Last time}
\label{sec:Last-time}

\begin{frame}{Last time}
	\begin{itemize}
		\item Spectral theory\vfill
		\item Finding eigenvalues\vfill
		\item Finding eigenspaces of eigenvectors
	\end{itemize}
\end{frame}

\section{Diagonalisation}

\begin{frame}
\begin{beamercolorbox}[sep=12pt,center]{part title}
\usebeamerfont{section title}
\insertsection\par
\end{beamercolorbox}
\end{frame}

\begin{frame}{Motivation}
If $x\in \mathbb{R}^n$ is a linear combination of eigenvectors of $A$ then:
\begin{equation*}
	A(x) = A\left(\sum_{i=1}^n a_i v_{\lambda_i}\right) = \sum_{i=1}^n a_iA(v_{\lambda_i}) = \sum_{i=1}^n a_i\lambda_iv_{\lambda_i}
\end{equation*}
So if \emph{every} vector in $\mathbb{R}^n$ can be written as a linear combination of eigenvectors of $A$ the the \emph{entire} matrix action simplifies.\vfill
\begin{definition}
	An $n\times n$ matrix $A$ is \emph{diagonalisable} iff every vector in $\mathbb{R}^n$ can be written as a linear combination of eigenvectors.
\end{definition}
\end{frame}

\begin{frame}{Example}
\begin{example}
	\begin{equation*}
	\left[
	\begin{matrix}
	3&0\\
	0&5
	\end{matrix}
	\right]
	\text{ has eigenvectors }
	\left[
	\begin{matrix}
	1\\
	0
	\end{matrix}
	\right]\text{ and }
	\left[
	\begin{matrix}
	0\\
	1
	\end{matrix}
	\right]
	\end{equation*}
	and so is diagonalisable.
\end{example}
\begin{example}
	\begin{equation*}
		A = \left[\begin{matrix}
			3 & -1\\
			-1 & 3
		\end{matrix}\right]
	\text{ has eigenvectors }
	v_2 = \left[
	\begin{matrix}
	1\\
	1
	\end{matrix}
	\right]\text{ and }
	v_4 = \left[
	\begin{matrix}
	1\\
	-1
	\end{matrix}
	\right]
	\end{equation*}
	and so is diagonalisable because all vectors in $\mathbb{R}^2$ can be written as a linear combination of $v_2$ and $v_4$.
\end{example}
\end{frame}

\begin{frame}[fragile]{Motivation}
Suppose we want to find $Av$.
{\LARGE
\begin{equation*}
\begin{tikzcd}[row sep = 2cm]
v = \sum_{i=1}^n v_i e_i \dar[mapsto]{P^{-1}} & Av = \sum_{i=1}^n (Av)_i e_i\\
\sum_{i=1}^n a_i v_{\lambda_i} \rar[mapsto]{\text{Apply }A} & \sum_{i=1}^n a_i\lambda_i v_{\lambda_i} \uar[mapsto][swap]{P}
\end{tikzcd}
\end{equation*}
}
\begin{itemize}
	\item where $P$ writes a linear combination of eigenvectors as a linear combination of standard basis vectors
	\item we say that the bottom horizontal action is \emph{diagonal}
\end{itemize}
\end{frame}

\begin{frame}{Diagonalisation}
\begin{definition}
The matrix $P$ \emph{diagonalises $A$} iff there is a diagonal matrix $D$ such that
\begin{equation*}
A = PDP^{-1}
\end{equation*}
\end{definition}\vfill
{\bf Question:} How do we find the matrix $P$ for a matrix $A$?\vfill
{\bf Answer:} If $v_{\lambda_1}$, $v_{\lambda_2}$ \dots $v_{\lambda_n}$ are eigenvectors such that every $x\in \mathbb{R}^n$ is a linear combination of the $v_{\lambda_i}$ then the matrix
\begin{equation*}
P = \left[ v_{\lambda_1} v_{\lambda_2} \dots v_{\lambda_n}\right]
\end{equation*}
with eigenvectors as the columns is a diagonalising matrix for $A$.
\end{frame}

\begin{frame}{When is a matrix diagonalisable?}
\begin{theorem}
The following are equivalent for a matrix $A$:
\begin{enumerate}
	\item there is a $P$ such that $P^{-1}AP$ is diagonal
	\item there are $n$ eigenvectors such that any $x\in \mathbb{R}^n$ is a linear combination of these eigenvectors
	\item there are eigenvectors $v_1\dots v_n$ such that $\left[v_1 v_2\dots v_n \right]$ is invertible
	\item for every eigenvalue $\lambda$ the geometric multiplicity is equal to the algebraic multiplicity
\end{enumerate}
\end{theorem}

\begin{lemma}
If the characteristic polynomial of $A$ has $n$ distinct eigenvalues then $A$ is diagonalisable by (4).
\end{lemma}
\end{frame}

\begin{frame}{Examples}
\begin{example}
If possible diagonalise
\begin{equation*}
\left[
\begin{matrix}
1&1\\
-1&1
\end{matrix}
\right]
\end{equation*}
\end{example}
\begin{example}
If possible diagonalise
\begin{equation*}
\left[
\begin{matrix}
1&1\\
0&1
\end{matrix}
\right]
\end{equation*}
\end{example}
\end{frame}

\begin{frame}{Examples}
\begin{example}
If possible diagonalise
\[
\left[
\begin{matrix}
3&-4&2\\
1&-2&2\\
1&-5&5
\end{matrix}
\right]\]
\end{example}
\begin{example}
If possible diagonalise
\[
\left[
\begin{matrix}
1&0&1\\
0&1&0\\
0&0&-3
\end{matrix}
\right]\]
\end{example}
\end{frame}


\begin{frame}{Summary}
\begin{enumerate}
	\item Find the eigenvalues by solving $\left|A-\lambda\cdot I\right|$.
	\begin{itemize}
		\item The no. of $(\lambda - \lambda_i)$ factors is the \emph{algebraic multiplicity $alg(\lambda_i)$}.
	\end{itemize}
	\vfill
	\item Find the eigenspaces $E_{\lambda_i}$ by solving $(A-\lambda_i\cdot I) = 0$.
	\begin{itemize}
		\item Only need to do this for the eigenvalues $\lambda_i$ found in (1).
		\item The no. of parameters is the \emph{geometric multiplicity $geom(\lambda_i)$}.
	\end{itemize}\vfill
	\item 
	\begin{itemize}
		\item If for all $\lambda_i$ found in (1) we have $geom(\lambda_i) = alg(\lambda_i)$
		then the diagonalising matrix is $P= \left[v_{\lambda_1} v_{\lambda_2} \dots v_{\lambda_n}\right]$
		where the $v_{\lambda_i}$ are the basic eigenvectors.
		\item If for any of the $\lambda_i$ found in (1) has $geom(\lambda_i) < alg(\lambda_i)$
		then the matrix is not diagonalisable.
	\end{itemize}
\end{enumerate}
\end{frame}

\section{Similar matrices}

\begin{frame}
\begin{beamercolorbox}[sep=12pt,center]{part title}
\usebeamerfont{section title}
\insertsection\par
\end{beamercolorbox}
\end{frame}

\begin{frame}{Similar matrices}
\begin{definition}
	Two matrices $A$ and $B$ are \emph{similar} iff there exists an invertible $P$ such that
	\begin{equation*}
	A = PBP^{-1}
	\end{equation*}
	and we write $A\sim B$.
\end{definition}
\end{frame}

\begin{frame}{Trace of a matrix}
\begin{definition}
	The trace of a matrix is the sum of its diagonal elements:
	\begin{equation*}
	tr(A) = \sum_{i=1}^n a_{ii}
	\end{equation*}
\end{definition}
\begin{theorem}
	If $A\sim B$ then $tr(A) = tr(B)$.
\end{theorem}
\end{frame}

\section{Powers}

\begin{frame}
\begin{beamercolorbox}[sep=12pt,center]{part title}
\usebeamerfont{section title}
\insertsection\par
\end{beamercolorbox}
\end{frame}


\begin{frame}{Taking powers of a diagonalisable matrix}
Suppose that there exists a diagonal matrix $D$ such that
\begin{equation*}
A = P D P^{-1}
\end{equation*}
for some invertible matrix $P$.
Then
\begin{equation*}
A^2 = PDP^{-1}PDP^{-1} = PD^2P^{-1}
\end{equation*}
and indeed
\begin{equation*}
A^n = PD^nP^{-1}
\end{equation*}
\end{frame}

\begin{frame}{Examples}
\begin{example}
	Find $A^9$ if
	\begin{equation*}
	A = \left[
	\begin{matrix}
	5&0\\
	0&2
	\end{matrix}
	\right]
	\end{equation*}
\end{example}
\begin{example}
	Find $A^{50}$ if 
	\begin{equation*}
	A = \left[
	\begin{matrix}
	2&1&0\\
	0&1&0\\
	-1&-1&1
	\end{matrix}
	\right]
	\end{equation*}
\end{example}
\end{frame}

\section{Dynamical systems}

\begin{frame}
\begin{beamercolorbox}[sep=12pt,center]{part title}
\usebeamerfont{section title}
\insertsection\par
\end{beamercolorbox}
\end{frame}

\begin{frame}{Markov chains}
\begin{definition}
	A \emph{dynamical system} consists of a function $\alpha(t)$ that prescribes how the state of the system changes over time. 
\end{definition}
\begin{definition}
	A \emph{discrete linear dynamical system} consists of a sequence of vectors
	\begin{equation*}
	x_0, x_1, x_2, \dots, x_k, \dots
	\end{equation*}
	such that $x_{k+1} = Ax_{k}$ for some matrix $A$.
\end{definition}
\end{frame}

\begin{frame}{Long term behaviour using eigenvectors}
If $x_0$ is a linear combination of the eigenvectors $v_{\lambda_i}$ of $A$ then 
\begin{equation*}
x_k = A^k x_0 = A^k\left(\sum_{i=1}^n b_i v_{\lambda_i}\right) = \sum_{i=1}^n b_i A^k\left(v_{\lambda_i}\right) = \sum_{i=1}^nb_i(\lambda_i)^kv_{\lambda_i}
\end{equation*}
and so the long-term behaviour is determined by the limits:
\begin{equation*}
\lim_{k\to \infty} (\lambda_i)^k
\end{equation*}
\end{frame}

\begin{frame}{Dominant eigenvalue}
\begin{definition}
	If a is a square matrix then a \emph{dominant eigenvalue $\lambda_{max}$} is one for which $|\lambda_{max}|>|\lambda_i|$ for all other eigenvalues $\lambda_i$.
\end{definition}
\begin{equation*}
x_k = \sum_{i=1}^n b_i (\lambda_i)^k v_{\lambda_i} \approx b_i (\lambda_{max})^k v_{\lambda_{max}}
\end{equation*}
and we can read off the long term behaviour. E.g.
\begin{itemize}
	\item if $|\lambda_{max}|< 1$ then the system converges to $0$
	\item if $|\lambda_{max}| = 1$ then the system converges to $b_iv_{\lambda_{max}}$
	\item if $|\lambda_{max}| = -1$ then the system oscillates between $\pm b_iv_{\lambda_{max}}$
	\item if $|\lambda_{max}>1|$ then the system diverges
\end{itemize}
\end{frame}

\begin{frame}{Examples}
\begin{example}
	Find a formula for $x_k$ if $x_{k+1} = Ax_k$,
	\begin{equation*}
	x_0 = \left[
	\begin{matrix}
	1\\
	-1
	\end{matrix}
	\right] \text{ and } A = \left[
	\begin{matrix}
	2&0\\
	3&-1
	\end{matrix}
	\right]
	\end{equation*}
\end{example}
\begin{example}
	Estimate the long term behaviour of the dynamical system with
	\begin{equation*}
	x_0 = \left[
	\begin{matrix}
	100\\
	40
	\end{matrix}
	\right] \text{ and }
	A = \left[
	\begin{matrix}
	\frac{1}{2}&\frac{1}{4}\\
	2&0
	\end{matrix}
	\right]
	\end{equation*}

\end{example}
\end{frame}

\section{Markov chains}

\begin{frame}
\begin{beamercolorbox}[sep=12pt,center]{part title}
\usebeamerfont{section title}
\insertsection\par
\end{beamercolorbox}
\end{frame}

\begin{frame}{Markov chains}
A \emph{Markov chain} consists of:-\vfill
\begin{itemize}
	\item a finite set of \emph{states} $x_1$, $x_2$, \dots, $x_n$\vfill
	\item a repeated \emph{transition interval} at the end of which the system transitions between states\vfill
	\item a \emph{non-deterministic rule} for predicting the probability that the system will transition into a certain state
	\begin{itemize}
		\item {\bf This probability only depends on the current state.}
		\item (Not the entire history of the chain.)
	\end{itemize}
\end{itemize}
\end{frame}

\begin{frame}{Markov transition matrices}
This means that a Markov chain is described by a matrix $A$ such that
\begin{align*}
A_{ij} &= \mathbb{P}(X_1 = j | X_0 = i)\\
&= \text{ the probability that the next state will be $j$ }\\
&\hspace{0.5cm} \text{ if the current state is $i$}
\end{align*}
Therefore:-
\begin{itemize}
	\item all of the entries are between $0$ and $1$
	\begin{itemize}
		\item (I.e. they are probabilities.)
	\end{itemize}
	\item in any column the sum of the entries is $1$
	\begin{itemize}
		\item (The system must transition into one of the states.)
	\end{itemize}
\end{itemize}
\end{frame}

\begin{frame}{Example}
\begin{example}
	Find the probability that $x_3$ is in state $1$ if
	\begin{equation*}
	x_0 = \left[
	\begin{matrix}
	0\\
	0\\
	1
	\end{matrix}
	\right] \text{ and }
	A = \left[
	\begin{matrix}
	0.4&0.25&0.2\\
	0.4&0.35&0.5\\
	0.2&0.4& 0.3
	\end{matrix}
	\right]
	\end{equation*}
\end{example}
\end{frame}

\end{document}