\documentclass{beamer}

\useoutertheme[subsection=false]{miniframes}
\usecolortheme{beaver}
\setbeamertemplate{navigation symbols}{}
\setbeamertemplate{footline}{}
\usepackage{graphicx}
\usepackage{url}
\usepackage{datetime}
\usepackage{tikz-cd}
\newcommand{\lectureDate}{\formatdate{30}{10}{2018}}

\setbeamertemplate{caption}
{\raggedright\insertcaption\par}
\title{MATH211: Linear Methods I}
\author{Matthew Burke}
\date{\lectureDate}
\begin{document}

\frame{\titlepage}

\begin{frame}{Lecture on \lectureDate}
  \tableofcontents
\end{frame}

\section*{Last time}
\label{sec:Last-time}

\begin{frame}{Last time}
  \begin{itemize}
  \item	Triangle inequality and Lagrange identity.\vfill
  \item More example problems in Euclidean space.\vfill
  \item Linear transformations.
  \end{itemize}
\end{frame}

\section{Linear transformations}

\begin{frame}
\begin{beamercolorbox}[sep=12pt,center]{part title}
\usebeamerfont{section title}
\insertsection\par
\end{beamercolorbox}
\end{frame}

\begin{frame}{Linear transformations}
\begin{definition}
A \emph{linear transformation $T:\mathbb{R}^n \rightarrow \mathbb{R}^m$} is a function satisfying:
\begin{itemize}
	\item (Additivity) $T(x+y) = T(x) + T(y)$
	\item (Scalar multiplication) $T(kx) = kT(x)$
\end{itemize}
\end{definition}
\end{frame}

\begin{frame}{Examples}
\begin{example}
Is the following a linear transformation?
\begin{equation*}
T \left[
\begin{array}{c}
x\\
y
\end{array}
\right] = \left[
\begin{array}{c}
2x\\
y\\
-x+2y
\end{array}
\right]
\end{equation*}
\end{example}
\begin{example}
Is the following a linear transformation?
\begin{equation*}
T \left[
\begin{array}{c}
x\\
y
\end{array}
\right] = \left[
\begin{array}{c}
xy\\
x+y
\end{array}
\right]
\end{equation*}
\end{example}
\end{frame}

\begin{frame}{Examples}
\begin{example}
Every matrix $A$ determines a linear transformation $x \mapsto Ax$.
\end{example}
\begin{example}
Is rotation about the origin in $\mathbb{R}^2$ through an angle $\theta$ a linear transformation?
(The standard convention is to measure the angle in the anti-clockwise direction.)
\end{example}
\begin{example}
Is reflection in the x-axis a linear transformation?
\end{example}
\begin{example}
Is $f(x) = x+1$ a linear transformation?
\end{example}
\end{frame}

\section{Action on combinations}

\begin{frame}
\begin{beamercolorbox}[sep=12pt,center]{part title}
\usebeamerfont{section title}
\insertsection\par
\end{beamercolorbox}
\end{frame}

\begin{frame}{Action on combinations}
Recall that if $T$ is linear then
\begin{equation*}
T\left(\sum_{i=0}^n k_i v_i\right) = \sum_{i=0}^n k_i T(v_i)
\end{equation*}
so if we know $T(v_i)$ then we can work out the value of $T$ on any linear combination of the $v_i$.
\end{frame}

\begin{frame}{Examples}
\begin{example}
	Find
	\begin{equation*}
	T \left[
	\begin{array}{c}
	-7\\
	3\\
	-9
	\end{array}
	\right]
	\end{equation*}
	if $T$ is linear and
	\begin{equation*}
	T \left[
	\begin{array}{c}
	1\\
	3\\
	1
	\end{array}
	\right] = \left[
	\begin{array}{c}
	4\\
	4\\
	0\\
	-2
	\end{array}
	\right]\text{ and } T \left[
	\begin{array}{c}
	4\\
	0\\
	5
	\end{array}
	\right] = \left[
	\begin{array}{c}
	4\\
	5\\
	-1\\
	5
	\end{array}
	\right]
	\end{equation*}
	% Ans: [-4,-6,2,-12]
\end{example}
\end{frame}


\begin{frame}{Examples}
\begin{example}
	Find
	\begin{equation*}
	T \left[
	\begin{array}{c}
	1\\
	3\\
	-2\\
	-4
	\end{array}
	\right]
	\end{equation*}
	if $T$ is linear and
	\begin{equation*}
	T \left[
	\begin{array}{c}
	1\\
	1\\
	0\\
	-2
	\end{array}
	\right] = \left[
	\begin{array}{c}
	2\\
	3\\
	-1
	\end{array}
	\right]\text{ and } T \left[
	\begin{array}{c}
	0\\
	-1\\
	1\\
	1
	\end{array}
	\right] = \left[
	\begin{array}{c}
	5\\
	0\\
	1
	\end{array}
	\right]
	\end{equation*}
	% Ans: [-8,3,-3]
\end{example}
\end{frame}

\section{Matrices redux}

\begin{frame}
\begin{beamercolorbox}[sep=12pt,center]{part title}
\usebeamerfont{section title}
\insertsection\par
\end{beamercolorbox}
\end{frame}

\begin{frame}{Linear transformations determined by action on basis}
If $T$ is linear then
\begin{equation*}
T\left(\sum_{i=0}^n k_i v_i\right) = \sum_{i=0}^n k_i T(v_i)
\end{equation*}
But for all $v$ in $\mathbb{R}^n$ we can write $v$ as a linear combination of the standard basis vectors:
\begin{equation*}
v = \left[
\begin{array}{c}
v_1\\
v_2\\
v_3\\
\vdots\\
v_n
\end{array}
\right] = \sum_{i=0}^n v_i e_i = v_1 \left[
\begin{array}{c}
1\\
0\\
0\\
\vdots\\
0
\end{array}
\right] + v_2 \left[
\begin{array}{c}
0\\
1\\
0\\
\vdots\\
0
\end{array}
\right] + v_3 \left[
\begin{array}{c}
0\\
0\\
1\\
\vdots\\
0
\end{array}
\right]+\dots v_n \left[
\begin{array}{c}
0\\
0\\
0\\
\vdots\\
1
\end{array}
\right]
\end{equation*}
\end{frame}

\begin{frame}{Linear transformation determined by action on basis}
So:
\begin{equation*}
T(v) = T\left(\sum_{i=0}^n v_i e_i\right) = \sum_{i=0}^n v_i T(e_i)
\end{equation*}\vfill
{\bf Slogan:} A linear transformation is completely determined by its action on the standard basis vectors.\vfill
More explicitly: we only need to know
\begin{equation*}
T(e_1), T(e_2),\dots, T(e_n)
\end{equation*}
to work out the whole transformation!
\end{frame}

\begin{frame}{Matrices redux}
If $T:\mathbb{R}^n \rightarrow \mathbb{R}^m$ then
\begin{equation*}
T(e_1), T(e_2),\dots, T(e_n)
\end{equation*}
are in $\mathbb{R}^m$. And so the transformation is completely determined by $n\times m$ scalars\dots
\end{frame}

\begin{frame}{Matrices redux}
\begin{definition}
The \emph{matrix $[T]$ of a linear transformation $T$} is the matrix that has columns given by the images of the standard basis vectors:
\begin{equation*}
[T] = \left[
\begin{array}{cccc}
T(e_1) & T(e_2) &\dots & T(e_n)
\end{array}
\right]
\end{equation*} 
\end{definition}\vfill
(It turns out that the definition of matrix product is forced on us after this choice if we want it to match the composition of functions.)
\end{frame}

\begin{frame}{Examples}
\begin{example}
Find the matrix of the linear transformation
\begin{equation*}
T \left[
\begin{array}{c}
x\\
y
\end{array}
\right] = \left[
\begin{array}{c}
x+2y\\
x-y
\end{array}
\right]
\end{equation*}
\end{example}
\begin{example}
Find the matrix of the linear transformation satisfying
\begin{equation*}
T \left[
\begin{array}{c}
1\\
1
\end{array}
\right] = \left[
\begin{array}{c}
1\\
2
\end{array}
\right]\text{ and } T \left[
\begin{array}{c}
0\\
-1
\end{array}
\right] = \left[
\begin{array}{c}
3\\
2
\end{array}
\right]
\end{equation*}
\end{example}
\end{frame}


\begin{frame}{Examples}
\begin{example}
What is the matrix of the linear transformation for counter-clockwise rotation about the origin by an angle of $\theta$?
\end{example}
\begin{example}
What is the matrix of the linear transformation for reflection in the x-axis?
\end{example}
\begin{example}
What is the matrix of the linear transformation that is $v\times (-)$ for some vector $v\in \mathbb{R}^3$?
\end{example}
\begin{example}
What is the matrix of the linear transformation for reflection in the line $y = mx$?
\end{example}
\end{frame}

\section{Composition}

\begin{frame}
\begin{beamercolorbox}[sep=12pt,center]{part title}
\usebeamerfont{section title}
\insertsection\par
\end{beamercolorbox}
\end{frame}

\begin{frame}{Composition of functions}
\begin{definition}
If we have functions
\begin{equation*}
\mathbb{R}^n \xrightarrow{f} \mathbb{R}^m \text{ and }\mathbb{R}^m \xrightarrow{g} \mathbb{R}^p
\end{equation*}
then the \emph{composite function $g\circ f$} is the function defined by
\begin{equation*}
(g\circ f)(x) = g(f(x))
\end{equation*}
\end{definition}
\begin{itemize}
	\item I.e. first apply $f$ to $x$ and then apply $g$ to the result.
	\item Note that $\mathbb{R}^n \xrightarrow{g\circ f} \mathbb{R}^p$.
\end{itemize}
\begin{example}
The composite of two linear transformations is again linear.
(Easy exercise.)
\end{example}
\end{frame}

\begin{frame}{The matrix of a composite transformation}
We know that the matrix of the composite $T\circ S$ is:
\begin{equation*}
\left[
\begin{array}{cccc}
T(S(e_1)) & T(S(e_2)) & \dots & T(S(e_n))
\end{array}
\right]
\end{equation*}
where $S: \mathbb{R}^n \rightarrow \mathbb{R}^m$ and $T: \mathbb{R}^m \rightarrow \mathbb{R}^p$.\vfill
{\bf Question:} What is this in terms of the matrices of $T$ and $S$?\vfill
\begin{theorem}
The matrix of a composite is the composite of the matrices:
\begin{equation*}
[T\circ S] = [T][S]
\end{equation*}
\begin{proof}
Note that the terms $S(e_i)$ are the columns of $[S]$.
\end{proof}
\end{theorem}
\end{frame}

\begin{frame}{Examples}
\begin{example}
What is the matrix of the linear transformation consisting of first reflects in the x-axis and then rotates through an angle of $\frac{\pi}{2}$?
\end{example}
\begin{example}
Find the rotation or reflection that equals reflection in the line $y = -x$ followed by reflection in the y-axis.
\end{example}
\end{frame}

\begin{frame}{Summary of linear transformations}
\begin{itemize}
	\item Recall interpretation of matrix action on a vector
	\item Definition of linear transformation; examples and non-examples
	\item The matrix of a linear transformation; completely determined by action on basis
	\item Linear transformation preserves 0, - and linear combinations
	\item Composition of functions, linear transformation and relationship to matrices.
	\item Inverse functions and relationship to inverse matrices.
	\item Determinant of rotation and reflection. Composite of rotations/reflections with rotations/reflections
\end{itemize}
\end{frame}

\begin{frame}{Examples}
\begin{example}
For the following linear transformations $S$ and $T$ what is $S\circ T$, $T\circ S$ and what are the corresponding matrices?
\begin{equation*}
S \left[
\begin{array}{c}
x\\
y
\end{array}
\right] = \left[
\begin{array}{c}
x\\
-y
\end{array}
\right]\text{ and } T \left[
\begin{array}{c}
x\\
y
\end{array}
\right] = \left[
\begin{array}{c}
-y\\
x
\end{array}
\right]
\end{equation*}
\end{example}
\begin{example}
What is the inverse transformation and corresponding matrix for
\begin{equation*}
T \left[
\begin{array}{c}
x\\
y
\end{array}
\right] = \left[
\begin{array}{c}
x+y\\
y
\end{array}
\right]
\end{equation*}
\end{example}
\end{frame}


\end{document}