\documentclass{beamer}

\useoutertheme[subsection=false]{miniframes}
\usecolortheme{beaver}
\setbeamertemplate{navigation symbols}{}
\setbeamertemplate{footline}{}
\usepackage{graphicx}
\usepackage{url}
\usepackage{datetime}
%\usepackage{enumitem}
% \SetLabelAlign{parright}{\parbox[t]{\labelwidth}{\raggedleft#1}}

% \setlist[description]{style=multiline,topsep=10pt,leftmargin=3cm,align=parright}

\newcommand {\framedgraphic}[3] {
  \begin{frame}{#1}
    \vspace{-0.5cm}
    \begin{center}
      \includegraphics[width=0.9\textwidth,keepaspectratio]{#2}
    \end{center}
    \vspace{-1cm}
    \begin{center}
      #3
    \end{center}
  \end{frame}
}
\setbeamertemplate{caption}{\raggedright\insertcaption\par}
\title{MATH211: Linear Methods I\\Lecture 2}
\author{Matthew Burke}
\date{\formatdate{11}{09}{2018}}
\begin{document}

\frame{\titlepage}

\begin{frame}{Lecture 2}
  \tableofcontents
\end{frame}

\section{Reduced row echelon form}
\label{sec:Reduced-row-echelon-form}

\begin{frame}
    \begin{beamercolorbox}[sep=12pt,center]{part title}
      \usebeamerfont{section title}\insertsection\par
    \end{beamercolorbox}
\end{frame}
\begin{frame}{Row echelon form}
  A matrix is in \emph{row echelon form} iff:-
  \begin{itemize}
  \item All rows consisting entirely of zeros are at the 
    bottom.
  \item The first nonzero entry in each nonzero row is a $1$
    \\(called the {\bf leading 1} for that row).
  \item Each leading 1 is to the right of all leading $1$'s 
    in rows above it.
  \end{itemize}
  For instance:
  \begin{equation*}
    \left[\begin{array}{rrrrrrrr}
            0 & 1 & * & * & * & * & * & * \\
            0 & 0 & 0 & 1 & * & * & * & * \\
            0 & 0 & 0 & 0 & 1 & * & * & * \\
            0 & 0 & 0 & 0 & 0 & 0 & 0 & 1 \\
            0 & 0 & 0 & 0 & 0 & 0 & 0 & 0 \\
            0 & 0 & 0 & 0 & 0 & 0 & 0 & 0 
          \end{array}\right]
      \end{equation*}
      where $*$ can be any number.
\end{frame}

\begin{frame}{Reduced row echelon form}
  A matrix is in \emph{reduced row echelon form} iff:-
  \begin{itemize}
  \item It is a row-echelon matrix.
  \item Each leading 1 is the only nonzero entry in its
    column.
  \end{itemize}
  \begin{equation*}
      \left[\begin{array}{rrrrrrrr}
              0 & 1 & * & 0 & 0 & * & * & 0 \\
              0 & 0 & 0 & 1 & 0 & * & * & 0 \\
              0 & 0 & 0 & 0 & 1 & * & * & 0 \\
              0 & 0 & 0 & 0 & 0 & 0 & 0 & 1 \\
              0 & 0 & 0 & 0 & 0 & 0 & 0 & 0 \\
              0 & 0 & 0 & 0 & 0 & 0 & 0 & 0 
            \end{array}\right] 
        \end{equation*}
        where $*$ can be any number.
\end{frame}

\begin{frame}{Advantages of reduced row echelon form}
  \begin{itemize}
  \item Row echelon form $\leftrightarrow$ ready for back-substitution.\vfill
  \item Reduced row echelon form $\leftrightarrow$ read off solutions.
  \end{itemize}\vfill
  Another advantage of reduced row echelon form is that:\vfill
  \begin{theorem}
    Reduced row echelon form is unique.
  \end{theorem}
\end{frame}

\begin{frame}{Reading off solutions}
  \begin{itemize}
  \item   The augmented matrix
    \begin{equation*}
      \left[\begin{array}{rrrrr|r}
               1 & 0 & 5 & 0 & 1 &1 \\
               0 & 1 & 1 & 0 & 2 &0 \\
               0 & 0 & 0 & 1 & 4 &0 \\
               0 & 0 & 0 & 0 & 0 &0\\
               0 & 0 & 0 & 0 & 0 &0
            \end{array}\right] 
        \end{equation*}
        is in reduced row echelon form.\vfill
      \item If there are any rows of the form $(0~ 0...0|1)$ then we know immediately that there are no solutions.
  \end{itemize}
\end{frame}

\begin{frame}{Identify leading 1s and parameter blocks}
  \begin{itemize}
  \item The leading $1$s are shown in green
    \begin{equation*}
      \left[\begin{array}{rrrrr|r}
               {\color{green}1} & 0 & {\color{blue}5} & 0 & {\color{red}1} &1 \\
               0 & {\color{green}1} & {\color{blue}1} & 0 & {\color{red}2} &0 \\
               0 & 0 & 0 & {\color{green}1} & {\color{red}4} &0 \\
               0 & 0 & 0 & 0 & 0 &0\\
               0 & 0 & 0 & 0 & 0 &0
            \end{array}\right] 
        \end{equation*}
        and the other non-zero numbers are shown in blue and red.\vfill
      \item Variables associated to the leading $1$s are the \emph{leading variables}.\vfill
      \item In this case $x_1$, $x_2$ and $x_4$ are leading and $x_3$ and $x_5$ are not.
  \end{itemize}
\end{frame}

\begin{frame}
  The non-leading variables can take any value so we assign parameters to them.\vfill
  In this case:
  \begin{align*}
    {\color{blue}x_3} &{\color{blue}= s}\\
    {\color{red}x_5} &{\color{red}= t}
  \end{align*}
  and then we solve for the leading variables in terms of these parameters:
  \begin{align*}
    {\color{green}x_1} &= 1-{\color{blue}5s}-{\color{red}t}\\
    {\color{green}x_2} &= -{\color{blue}s}-{\color{red}2t}\\
    {\color{green}x_4} &= -{\color{blue}4t}
  \end{align*}
\end{frame}

\begin{frame}
  Questions?
\end{frame}


\section{Examples}
\subsection{}

\begin{frame}
    \begin{beamercolorbox}[sep=12pt,center]{part title}
      \usebeamerfont{section title}\insertsection\par
    \end{beamercolorbox}
\end{frame}

\begin{frame}{Linear equations in one variable}
  Find all real numbers $x$ such that
  \begin{equation*}
    ax = b
  \end{equation*}
  where $a$ and $b$ are real numbers.
\end{frame}

\begin{frame}{Linear equation in two variables}
  Find all real numbers $x$ and $y$ such that both
  \begin{align*}
    x+2y&=1\\
    3x+4y&=0
  \end{align*}
  hold.
\end{frame}

\begin{frame}{Elementary row operations}
  But what operations are we allowed to do on the rows?\vfill
  The following operations will not change the solutions:-\vfill
  \begin{enumerate}
  \item Swap two rows.\vfill
  \item Add a multiple of one row to another row.\vfill
  \item Multiply a row by a \emph{non-zero} scalar.\vfill
  \end{enumerate}
  See pictures on Jupyter notebook.
\end{frame}


\begin{frame}
  Questions?
\end{frame}


\begin{frame}{Inconsistent equations}
  Find all real numbers $x$ and $y$ such that both
  \begin{align*}
    x+2y&=1\\
    5x+10y&=42
  \end{align*}
  hold.
\end{frame}

\begin{frame}{Infinitely many solutions}
  Find all real numbers $x$ and $y$ such that both
  \begin{align*}
    3x+12y&=18\\
    4x+16y&=24
  \end{align*}
  hold.
\end{frame}

\begin{frame}
Questions?
\end{frame}

\section{Rank}
\subsection{}
\begin{frame}
    \begin{beamercolorbox}[sep=12pt,center]{part title}
      \usebeamerfont{section title}\insertsection\par
    \end{beamercolorbox}
\end{frame}

\begin{frame}{Definition of rank}
  \begin{definition}
    The \emph{rank} of a matrix is the number of leading $1$s in its reduced row echelon form.
  \end{definition}
  \begin{example}
    Therefore the rank of
    \begin{equation*}
      \left[\begin{array}{rrrrr|r}
              {\color{green}1} & 0 & {\color{blue}5} & 0 & {\color{red}1} &1 \\
              0 & {\color{green}1} & {\color{blue}1} & 0 & {\color{red}2} &0 \\
              0 & 0 & 0 & {\color{green}1} & {\color{red}4} &0 \\
              0 & 0 & 0 & 0 & 0 &0\\
              0 & 0 & 0 & 0 & 0 &0
            \end{array}\right] 
        \end{equation*}
        is $3$.
      \end{example}
\end{frame}

\begin{frame}
  Find all $x$, $y$ and $z$ such that
  \begin{align*}
    x+y+2z&=-1\\
    2x+y+3z&=0\\
    0x+-2y+1z&=2
  \end{align*}
  hold.
\end{frame}

\begin{frame}
  Find all $x$, $y$ and $z$ such that
  \begin{align*}
    x+2y+3z&=4\\
    5x+6y+7z&=8\\
    3x+4y+5z&=1
  \end{align*}
  hold.
\end{frame}

\begin{frame}
  Find all $x$, $y$ and $z$ such that
  \begin{align*}
    6x+4y+2z&=4\\
    3x+2y+1z&=2\\
    9x+6y+3z&=6
  \end{align*}
  hold.
\end{frame}

\begin{frame}
  Questions?
\end{frame}

\section{Examples}
\label{sec:Examples-2}

\begin{frame}{Example}
  test
\end{frame}

\end{document}