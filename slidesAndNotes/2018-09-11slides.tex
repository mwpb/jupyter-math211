\documentclass{beamer}

\useoutertheme[subsection=false]{miniframes}
\usecolortheme{beaver}
\setbeamertemplate{navigation symbols}{}
\setbeamertemplate{footline}{}
\usepackage{graphicx}
\usepackage{url}
\usepackage{datetime}
%\usepackage{enumitem}
% \SetLabelAlign{parright}{\parbox[t]{\labelwidth}{\raggedleft#1}}

% \setlist[description]{style=multiline,topsep=10pt,leftmargin=3cm,align=parright}

\newcommand {\framedgraphic}[3] {
  \begin{frame}{#1}
    \vspace{-0.5cm}
    \begin{center}
      \includegraphics[width=0.9\textwidth,keepaspectratio]{#2}
    \end{center}
    \vspace{-1cm}
    \begin{center}
      #3
    \end{center}
  \end{frame}
}
\setbeamertemplate{caption}{\raggedright\insertcaption\par}
\title{MATH211: Linear Methods I\\Lecture 2}
\author{Matthew Burke}
\date{\formatdate{11}{09}{2018}}
\begin{document}

\frame{\titlepage}

\begin{frame}{Lecture 2}
  \tableofcontents
\end{frame}

\section{Last time}
\label{sec:Last-time}

\begin{frame}{Last time}
  \begin{itemize}
  \item Linear equations in one variable\vfill
  \item Linear equations in two variables
    \begin{itemize}
    \item graphical interpretation
    \end{itemize}\vfill
  \item What elementary row operations are allowed?
    \begin{itemize}
    \item graphical interpretation
    \end{itemize}\vfill
  \item Linear equations in three variables\vfill
  \end{itemize}
\end{frame}

\section{Reduced row echelon form}
\label{sec:Reduced-row-echelon-form}

\begin{frame}
    \begin{beamercolorbox}[sep=12pt,center]{part title}
      \usebeamerfont{section title}\insertsection\par
    \end{beamercolorbox}
\end{frame}
\begin{frame}{Row echelon form}
  A matrix is in \emph{row echelon form} iff:-
  \begin{itemize}
  \item All rows consisting entirely of zeros are at the 
    bottom.
  \item The first nonzero entry in each nonzero row is a $1$
    \\(called the {\bf leading 1} for that row).
  \item Each leading 1 is to the right of all leading $1$'s 
    in rows above it.
  \end{itemize}
  For instance:
  \begin{equation*}
    \left[\begin{array}{rrrrrrrr}
            0 & 1 & * & * & * & * & * & * \\
            0 & 0 & 0 & 1 & * & * & * & * \\
            0 & 0 & 0 & 0 & 1 & * & * & * \\
            0 & 0 & 0 & 0 & 0 & 0 & 0 & 1 \\
            0 & 0 & 0 & 0 & 0 & 0 & 0 & 0 \\
            0 & 0 & 0 & 0 & 0 & 0 & 0 & 0 
          \end{array}\right]
      \end{equation*}
      where $*$ can be any number.
\end{frame}

\begin{frame}{Reduced row echelon form}
  A matrix is in \emph{reduced row echelon form} iff:-
  \begin{itemize}
  \item It is a row-echelon matrix.
  \item Each leading 1 is the only nonzero entry in its
    column.
  \end{itemize}
  \begin{equation*}
      \left[\begin{array}{rrrrrrrr}
              0 & 1 & * & 0 & 0 & * & * & 0 \\
              0 & 0 & 0 & 1 & 0 & * & * & 0 \\
              0 & 0 & 0 & 0 & 1 & * & * & 0 \\
              0 & 0 & 0 & 0 & 0 & 0 & 0 & 1 \\
              0 & 0 & 0 & 0 & 0 & 0 & 0 & 0 \\
              0 & 0 & 0 & 0 & 0 & 0 & 0 & 0 
            \end{array}\right] 
        \end{equation*}
        where $*$ can be any number.
\end{frame}

\begin{frame}{Advantages of reduced row echelon form}
  \begin{itemize}
  \item Row echelon form $\leftrightarrow$ ready for back-substitution.\vfill
  \item Reduced row echelon form $\leftrightarrow$ read off solutions.
  \end{itemize}\vfill
  Another advantage of reduced row echelon form is that:\vfill
  \begin{theorem}
    Reduced row echelon form is unique.
  \end{theorem}
\end{frame}

\begin{frame}{Reading off solutions}
  \begin{itemize}
  \item   The augmented matrix
    \begin{equation*}
      \left[
        \begin{array}{rrrrr|r}
          1 & 0 & 5 & 0 & 1 &1 \\
          0 & 1 & 1 & 0 & 2 &0 \\
          0 & 0 & 0 & 1 & 4 &0 \\
          0 & 0 & 0 & 0 & 0 &0\\
          0 & 0 & 0 & 0 & 0 &0
        \end{array}\right] 
    \end{equation*}
    is in reduced row echelon form.\vfill
  \item If there are any rows of the form $(0~ 0...0|1)$ then we know immediately that there are no solutions.
  \end{itemize}
\end{frame}

\begin{frame}{Identify leading 1s and parameter blocks}
  \begin{itemize}
  \item The leading $1$s are shown in green
    \begin{equation*}
      \left[\begin{array}{rrrrr|r}
               {\color{green}1} & 0 & {\color{blue}5} & 0 & {\color{red}1} &1 \\
               0 & {\color{green}1} & {\color{blue}1} & 0 & {\color{red}2} &0 \\
               0 & 0 & 0 & {\color{green}1} & {\color{red}4} &0 \\
               0 & 0 & 0 & 0 & 0 &0\\
               0 & 0 & 0 & 0 & 0 &0
            \end{array}\right] 
        \end{equation*}
        and the other non-zero numbers are shown in blue and red.\vfill
      \item Variables associated to the leading $1$s are the \emph{leading variables}.\vfill
      \item In this case $x_1$, $x_2$ and $x_4$ are leading and $x_3$ and $x_5$ are not.
  \end{itemize}
\end{frame}

\begin{frame}
  The non-leading variables can take any value so we assign parameters to them.\vfill
  In this case:
  \begin{align*}
    {\color{blue}x_3} &{\color{blue}= s}\\
    {\color{red}x_5} &{\color{red}= t}
  \end{align*}
  and then we solve for the leading variables in terms of these parameters:
  \begin{align*}
    {\color{green}x_1} &= 1-{\color{blue}5s}-{\color{red}t}\\
    {\color{green}x_2} &= -{\color{blue}s}-{\color{red}2t}\\
    {\color{green}x_4} &= -{\color{blue}4t}
  \end{align*}
\end{frame}

\begin{frame}
  Questions?
\end{frame}


\section{Examples}
\subsection{}

\begin{frame}
    \begin{beamercolorbox}[sep=12pt,center]{part title}
      \usebeamerfont{section title}\insertsection\par
    \end{beamercolorbox}
\end{frame}

\begin{frame}{Worked example}
  Solve
  \begin{equation*}
      \left[
        \begin{array}{rrrrr|r}
          0 & 0 & 0 & -2 & -8 & 4 \\
          -3 & 6 & -4 & 9 & 3 & -1 \\
          -1 & 2 & -2 & -4 & -3 & 3 \\
          1 & -2 & 1 & 3 & -1 & 1
        \end{array}\right] 
    \end{equation*}
\end{frame}

\begin{frame}{Infinitely many solutions}
  Solve
  \begin{equation*}
      \left[
        \begin{array}{rrrr|r}
          1 & 3 & 4 & 9 & 1 \\
          5 & 6 & 7 & 8 & 5 \\
          6 & 18 & 24 & 54 & 12 \\
          1 & 1 & 1 & 1 & 1
        \end{array}\right] 
    \end{equation*}
\end{frame}

\begin{frame}{Inconsistent}
  Solve
  \begin{equation*}
      \left[
        \begin{array}{rrrr|r}
          1 & 3 & 1 & 2 & 1 \\
          5 & 6 & 7 & 8 & 5 \\
          6 & 9 & 8 & 10 & 10 \\
          1 & 1 & 1 & 1 & 1
        \end{array}
      \right] 
    \end{equation*}
\end{frame}

\begin{frame}{Unique solution}
  Solve
  \begin{equation*}
      \left[
        \begin{array}{rrrr|r}
          1 & 2 & 3 & 4 & 1 \\
          5 & 6 & 7 & 8 & 5 \\
          11 & 10 & 10 & 12 & 6 \\
          1 & 1 & 1 & 4 & 1
        \end{array}
      \right] 
    \end{equation*}
\end{frame}

\begin{frame}
  Questions?
\end{frame}

\section{Terminology}
\subsection{}
\begin{frame}
    \begin{beamercolorbox}[sep=12pt,center]{part title}
      \usebeamerfont{section title}\insertsection\par
    \end{beamercolorbox}
\end{frame}

\begin{frame}{Gaussian elimination}
  An \emph{elementary row operation} is one of:-
  \begin{itemize}
  \item swap two rows
  \item multiply a row by a non-zero scalar
  \item add a multiple of one row to another
  \end{itemize}\vfill
  The method of \emph{Gaussian elimination} consists of:-
  \begin{itemize}
  \item using elementary row operations to reduce a matrix to reduced row echelon form
  \item make the non-leading variables parameters
  \item read off the solutions
  \end{itemize}
\end{frame}

\begin{frame}{Definition of rank}
  \begin{definition}
    The \emph{rank} of a matrix is the number of leading $1$s in its reduced row echelon form.
  \end{definition}
  \begin{example}
    Therefore the rank of the following matrix is $3$.
    \begin{equation*}
      \left[\begin{array}{rrrrr}
              1 & 0 & 5 & 0 & 1  \\
              0 & 1 & 1 & 0 & 2  \\
              0 & 0 & 0 & 1 & 4 \\
              0 & 0 & 0 & 0 & 0 \\
              0 & 0 & 0 & 0 & 0 
            \end{array}\right] 
        \end{equation*}
      \end{example}
      The rank is a property of a matrix without augmentation.
\end{frame}

\begin{frame}{Relationship to solution}
  For a \emph{consistent} system in $n$ variables of rank $r$:
  \begin{itemize}
  \item If $n=r$ there is a unique solution.
  \item If $r<n$ there are infinitely many solutions.
  \end{itemize}
\end{frame}

\begin{frame}{Homogeneous systems}
  A system of equations
  \begin{equation*}
      \left[
        \begin{array}{rrrr|r}
          a_{11} & a_{12} & ... & a_{1m} & b_1\\
          a_{21} & a_{22} & ... & a_{2m} & b_2\\
          ... & ... & ... & ... & ...\\
          a_{n1} & a_{n2} & ... & a_{nm} & b_n
        \end{array}
      \right] 
    \end{equation*}
    is \emph{homogeneous} iff for all $i$ we have $b_i = 0$.
    I.e. one that looks like:
    \begin{equation*}
      \left[
        \begin{array}{rrrr|r}
          a_{11} & a_{12} & ... & a_{1m} & 0\\
          a_{21} & a_{22} & ... & a_{2m} & 0\\
          ... & ... & ... & ... & ...\\
          a_{n1} & a_{n2} & ... & a_{nm} & 0
        \end{array}
      \right] 
    \end{equation*}
\end{frame}

\begin{frame}{Linear combinations}
  \begin{definition}
    A \emph{linear combination} of the vectors $v_1$, $v_2$, ..., $v_n$ is a sum of the form
    \begin{equation*}
      a_1v_1+a_2v_2+...+a_nv_n
    \end{equation*}
    where the $a_i$ are scalars.
  \end{definition}
\end{frame}

\begin{frame}{Example linear combinations}
  \begin{example}
    Suppose that we found that the solution to a system of equations was
    \begin{align*}
   x_1 &= 1-5s-t\\
    x_2 &= -s-2t\\
    x_4 &= -4t
    \end{align*}
    in parameters $s$ and $t$.
    Then the solutions are all linear combinations of the form
    \begin{equation*}
      \left[\begin{array}{r}
        1\\
        0\\
        0
      \end{array}\right]+s
      \left[\begin{array}{r}
        -5\\
        -1\\
        0
      \end{array}\right]+t
      \left[\begin{array}{r}
        -1\\
        -2\\
        -4
      \end{array}\right]
    \end{equation*}
  \end{example}
\end{frame}

\begin{frame}
  Questions?
\end{frame}

\section{Examples}
\label{sec:Examples-2}

\begin{frame}{Reading off example}
  Consider the system:
  \begin{equation*}
    \left[
      \begin{array}{rrrrr|r}
        1 & 0 & -3 & 0 & 6 & 8  \\
        0 & 1 & 1 & 0 & 2 & 0  \\
        0 & 0 & 0 & 1 & 5 & -5 \\
        0 & 0 & 0 & 0 & 0 & 0\\
        0 & 0 & 0 & 0 & 0 &0
      \end{array}\right] 
  \end{equation*}
  What is the rank of the corresponding matrix?
  What is the number of parameters?
  Is the system homogeneous?
\end{frame}

\begin{frame}{Worked example}
  Consider the system
  \begin{equation*}
    \left[
      \begin{array}{rrrr|r}
        1 & 0 & -1 & 0 & 0 \\
        2 & 0 & 0 & -1 & 0  \\
        0 & 2 & 0 & -2 & 0
      \end{array}\right] 
  \end{equation*}
  What is the rank of the corresponding matrix?
  What is the number of parameters?
  Is the system homogeneous?
\end{frame}

\begin{frame}{Computer example}
  Consider the system
  \begin{equation*}
    \left[
      \begin{array}{rrr|r}
        0 & 1 & 3 & 2  \\
        0 & 0 & 5 & 6   \\
        1 & 5 & 1 & -5 
      \end{array}\right] 
  \end{equation*}
  What is the rank of the corresponding matrix?
  What is the number of parameters?
  Is the system homogeneous?
\end{frame}

\begin{frame}{Computer example}
  Consider the system
  \begin{equation*}
    \left[
      \begin{array}{rrr|r}
        9 & -1 & 0 & -24  \\
        -1 & 7 & -4 & 17   \\
        0 & -4 & 8 & -14 
      \end{array}\right] 
  \end{equation*}
  What is the rank of the corresponding matrix?
  What is the number of parameters?
  Is the system homogeneous?
\end{frame}

\begin{frame}
  Questions?
\end{frame}

\end{document}