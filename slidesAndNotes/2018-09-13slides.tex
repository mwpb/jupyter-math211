\documentclass{beamer}

\useoutertheme[subsection=false]{miniframes}
\usecolortheme{beaver}
\setbeamertemplate{navigation symbols}{}
\setbeamertemplate{footline}{}
\usepackage{graphicx}
\usepackage{url}
\usepackage{datetime}

\newcommand {\framedgraphic}[3] {
  \begin{frame}{#1}
    \vspace{-0.5cm}
    \begin{center}
      \includegraphics[width=0.9\textwidth,keepaspectratio]{#2}
    \end{center}
    \vspace{-1cm}
    \begin{center}
      #3
    \end{center}
  \end{frame}
}
\newcommand{\lectureDate}{\formatdate{13}{09}{2018}}

\setbeamertemplate{caption}{\raggedright\insertcaption\par}
\title{MATH211: Linear Methods I}
\author{Matthew Burke}
\date{\formatdate{13}{09}{2018}}
\begin{document}

\frame{\titlepage}

\begin{frame}{Lecture on \formatdate{13}{09}{2018}}
  \tableofcontents
\end{frame}

\section{Last time}
\label{sec:Last-time}

\begin{frame}{Last time}
  \begin{itemize}
  \item Reduced row echelon form\vfill
  \item Rank\vfill
  \item Homogeneous systems\vfill
  \item Linear combinations\vfill
  \item Examples
  \end{itemize}
\end{frame}

\section{Fundamentals of matrices}

\begin{frame}
  \begin{beamercolorbox}[sep=12pt,center]{part title}
    \usebeamerfont{section title}\insertsection\par
  \end{beamercolorbox}
\end{frame}

\begin{frame}{Definition}
  \begin{definition}
    An \emph{$m\times n$ matrix} is a rectangular array of numbers with $m$ rows and $n$ columns:
    \begin{equation*}
      A=\left[
        \begin{array}{ccccc}
          a_{11} & a_{12} & a_{13} & \ldots & a_{1n} \\
          a_{21} & a_{22} & a_{23} & \ldots & a_{2n} \\
          a_{31} & a_{32} & a_{33} & \ldots & a_{3n} \\
          \vdots & \vdots & \vdots & & \vdots \\
          a_{m1} & a_{m2} & a_{m3} & \ldots & a_{mn} \\
        \end{array}
      \right] = \left[ a_{ij} \right]
    \end{equation*}
    We call the entry in the $i$th row and $j$th column $a_{ij}$.
  \end{definition}
\end{frame}

\begin{frame}{Special matrices}
  \begin{itemize}
  \item If $n=m$ we say the matrix is \emph{square}. E.g. if $n=m=2$:
    \begin{equation*}
      A=\left[
        \begin{array}{cc}
          a_{11} & a_{12} \\
          a_{21} & a_{22}
        \end{array}
      \right]
    \end{equation*}\vfill
  \item If $m=1$ we have a row vector: $X= \left[\begin{array}{cccc} a_{11} & a_{12} & \cdots & a_{1n}\end{array} \right]$\vfill
  \item If $n=1$ we have a column vector: $Y= \left[\begin{array}{c} a_{11} \\ a_{21} \\ \vdots \\ a_{m1}\end{array} \right]$
  \end{itemize}
\end{frame}

\begin{frame}{Matrix addition and subtraction}
  To add two matrices we add the corresponding entries.
  E.g.
  \begin{equation*}
      \left[
        \begin{array}{cc}
          1 & 3 \\
          2 & 3
        \end{array}
      \right]+
      \left[
        \begin{array}{cc}
          4 & 4 \\
          1 & 2
        \end{array}
      \right] =
      \left[
        \begin{array}{cc}
          5 & 7 \\
          3 & 5
        \end{array}
      \right]
    \end{equation*}\vfill
    In general if $A = A_{ij}$ and $B = B_{ij}$ are matrices then
    \begin{equation*}
      (A + B)_{ij} = A_{ij}+B_{ij}
    \end{equation*}\vfill
    Similarly for matrix subtraction, we subtract the corresponding entries;
    \begin{equation*}
      (A-B)_{ij} = A_{ij}-B_{ij}
    \end{equation*}
\end{frame}

\begin{frame}{Scalar multiplication}
  If $A = A_{ij}$ is a matrix and $k$ is a number then the matrix $kA$ has entries
  \begin{equation*}
    (kA)_{ij} = k\cdot A_{ij}
  \end{equation*}\vfill
  I.e. we multiply every entry of the matrix by $k$. E.g.\vfill
  \begin{equation*}
    5\cdot \left[
        \begin{array}{cc}
          1 & 3 \\
          2 & 3
        \end{array}
      \right]=
      \left[
        \begin{array}{cc}
          5 & 15 \\
          10 & 15
        \end{array}
      \right]
  \end{equation*}
\end{frame}

\begin{frame}
  Questions?
\end{frame}

\section{Systems of equations}

\begin{frame}
  \begin{beamercolorbox}[sep=12pt,center]{part title}
    \usebeamerfont{section title}\insertsection\par
  \end{beamercolorbox}
\end{frame}

\begin{frame}{Matrix acting on vector}
  If
  \begin{equation*}
      A=\left[
        \begin{array}{ccccc}
          a_{11} & a_{12} & a_{13} & \ldots & a_{1n} \\
          a_{21} & a_{22} & a_{23} & \ldots & a_{2n} \\
          a_{31} & a_{32} & a_{33} & \ldots & a_{3n} \\
          \vdots & \vdots & \vdots & & \vdots \\
          a_{m1} & a_{m2} & a_{m3} & \ldots & a_{mn} \\
        \end{array}
      \right]
    \end{equation*}
  Then define the action of $A$ on a vector of length $n$ as:
  \begin{equation*}
    A\left[\begin{array}{c} x_1 \\ x_2 \\ \vdots \\ x_n\end{array} \right] = x_1\cdot \left[\begin{array}{c} a_{11} \\ a_{21} \\ \vdots \\a_{m1}\end{array} \right]+x_2\cdot \left[\begin{array}{c} a_{12} \\ a_{22} \\ \vdots \\a_{m2}\end{array} \right]+ x_3\left[\begin{array}{c} a_{13} \\ a_{23} \\ \vdots \\a_{m3}\end{array} \right] + \dots + x_n\left[\begin{array}{c} a_{1n} \\ a_{2n} \\ \vdots \\a_{mn}\end{array} \right]
  \end{equation*}
  \bf{This does not make sense if vector does not have length $n$!}
\end{frame}


\begin{frame}{Slogan}
  The vector
  \begin{equation*}
    A \left[\begin{array}{c} x_1 \\ x_2 \\ \vdots \\ x_n\end{array} \right]
  \end{equation*}
  is the linear combination of the columns of $A$ weighted by the $x_i$.
\end{frame}

\begin{frame}{Matrix acting on vector}
  \begin{example}
    \begin{equation*}
      \left[\begin{array}{cc}
          1 & 2 \\
          4 & 3
        \end{array}\right]
      \left[\begin{array}{c}
          5 \\
          6
        \end{array}\right]
    \end{equation*}
  \end{example}
  \begin{example}
    \begin{equation*}
      \left[\begin{array}{ccc}
          1 & 2 & 8 \\
          4 & 3 & -1
        \end{array}\right]
      \left[\begin{array}{c}
          5 \\
          6 \\
          2
        \end{array}\right]
    \end{equation*}
  \end{example}
\end{frame}

\begin{frame}{Matrix form of system of linear equations}
  Since
  \begin{equation*}
    A\left[\begin{array}{c} x_1 \\ x_2 \\ \vdots \\ x_n\end{array} \right] = x_1\cdot \left[\begin{array}{c} a_{11} \\ a_{21} \\ \vdots \\a_{m1}\end{array} \right]+x_2\cdot \left[\begin{array}{c} a_{12} \\ a_{22} \\ \vdots \\a_{m2}\end{array} \right]+ x_3\left[\begin{array}{c} a_{13} \\ a_{23} \\ \vdots \\a_{m3}\end{array} \right] + \dots + x_n\left[\begin{array}{c} a_{1n} \\ a_{2n} \\ \vdots \\a_{mn}\end{array} \right]
  \end{equation*}
  we can write a system of $m$ linear equations in $n$ variables as 
  \begin{equation*}
    A \left[\begin{array}{c} x_1 \\ x_2 \\ \vdots \\ x_n\end{array} \right] = \left[\begin{array}{c} b_1 \\ b_2 \\ \vdots \\ b_m\end{array} \right]
  \end{equation*}
  where $A$ is an $m\times n$ matrix. Examples follow..
\end{frame}

\begin{frame}{Example of matrix form}
  \begin{example}
    Find the matrix equation for the system
    \begin{align*}
      x+3y &= 9\\
      2x+4y &= 3
    \end{align*}
  \end{example}
\end{frame}

\begin{frame}{Example of linear combinations}
  \begin{example}
    Express
    \begin{equation*}
      b = \left[\begin{array}{c}
          1 \\
                  1 \\
                  1
        \end{array}\right]
    \end{equation*}
    as a linear combination of
    \begin{equation*}
      a_1 = \left[
        \begin{array}{c}
          1 \\
          2 \\
          3
        \end{array}
      \right],
      a_2 = \left[
        \begin{array}{c}
          0 \\
          -1 \\
          1
        \end{array}\right],
      a_3 = \left[
        \begin{array}{c}
          2 \\
          0 \\
          3
        \end{array}
      \right]\text{ and }
      a_4 = \left[
        \begin{array}{c}
          -1 \\
          1 \\
          1
        \end{array}\right]
    \end{equation*}
  \end{example}
\end{frame}

\begin{frame}
  Questions?
\end{frame}

\section{Composition}

\begin{frame}
  \begin{beamercolorbox}[sep=12pt,center]{part title}
    \usebeamerfont{section title}\insertsection\par
  \end{beamercolorbox}
\end{frame}

\begin{frame}{Matrix product}
  Let $A= a_{ij}$ be an $m\times n$ matrix and $B=b_{jk}$ be an $p\times m$ matrix.\vfill
  \begin{definition}
    The \emph{matrix product $BA$} is the matrix with columns
    \begin{equation*}
      \left[
      \begin{array}{ccccc}
        B\left[\begin{array}{c} a_{11} \\ a_{21} \\ \vdots \\a_{m1}\end{array} \right] B\left[\begin{array}{c} a_{12} \\ a_{22} \\ \vdots \\a_{m2}\end{array} \right] \dots B\left[\begin{array}{c} a_{1n} \\ a_{2n} \\ \vdots \\a_{mn}\end{array} \right]
      \end{array}\right]
    \end{equation*}
  \end{definition}\vfill
  In other words we apply the matrix $B$ to each of the columns of $A$.
\end{frame}

\begin{frame}{When does this make sense?}
Recall that
    \begin{equation*}
      A \left[\begin{array}{c} x_1 \\ x_2 \\ \vdots \\ x_n\end{array} \right]
    \end{equation*}
    doesn't make sense if $A$ doesn't have precisely $n$ columns.\vfill
   Therefore the matrix product $BA$ only makes sense iff
    \begin{equation*}
      \text{Number of rows}(A) = \text{Number of columns}(B)
    \end{equation*}
\end{frame}

\begin{frame}{Examples}
  \begin{example}
    Find $BA$ where
    \begin{equation*}
      B = \left[
        \begin{array}{ccc}
          -1 &0 & 3\\
          2 & -1 & 1
        \end{array}\right]
      \text{ and } A = \left[
        \begin{array}{ccc}
          -1 &1 & 2\\
          0 & -2 & 4\\
          1 & 0 & 0
        \end{array}\right]
    \end{equation*}
  \end{example}
  \begin{example}
    The matrix product $AB$ where
    \begin{equation*}
      A = \left[
        \begin{array}{ccc}
          -1 &1 & 2\\
          0 & -2 & 4\\
          1 & 0 & 0
        \end{array}\right]
      \text{ and }B = \left[
        \begin{array}{ccc}
          -1 &0 & 3\\
          2 & -1 & 1
        \end{array}\right]
    \end{equation*}
    does not make sense.
  \end{example}
\end{frame}

\begin{frame}
  Questions?
\end{frame}

\section{Non-commutativity}

\begin{frame}
  \begin{beamercolorbox}[sep=12pt,center]{part title}
    \usebeamerfont{section title}\insertsection\par
  \end{beamercolorbox}
\end{frame}

\begin{frame}{Warning}
  In general
  \begin{equation*}
    AB \neq BA
  \end{equation*}\vfill
  In other words the order matters.\vfill
  (Explanation in terms of linear transformations.)
\end{frame}

\begin{frame}{Examples}
  Four different examples on Jupyter of matrices $A$ and $B$ such that:\vfill
  \begin{enumerate}
  \item $AB$ exists but $BA$ does not.\vfill
  \item Both $AB$ and $BA$ exist but are of different sizes.\vfill
  \item Both $AB$ and $BA$ exist and are the same size, but they are different.\vfill
  \item $AB = BA$ (it does happen sometimes!)
  \end{enumerate}
\end{frame}

\begin{frame}
  Questions?
\end{frame}

\end{document}