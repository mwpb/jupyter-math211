\documentclass{beamer}

\useoutertheme[subsection=false]{miniframes}
\usecolortheme{beaver}
\setbeamertemplate{navigation symbols}{}
\setbeamertemplate{footline}{}
\usepackage{graphicx}
\usepackage{url}
\usepackage{datetime}
\usepackage{tikz-cd}
\newcommand{\lectureDate}{\formatdate{04}{12}{2018}}

\setbeamertemplate{caption}
{\raggedright\insertcaption\par}
\title{MATH211: Linear Methods I}
\author{Matthew Burke}
\date{\lectureDate}
\begin{document}

\frame{\titlepage}

\begin{frame}{Lecture on \lectureDate}
  \tableofcontents
\end{frame}

\section*{Last time}
\label{sec:Last-time}

\begin{frame}{Last time}
	\begin{itemize}
		\item Diagonalisation\vfill
		\item Matrix powers
	\end{itemize}
\end{frame}

\section{Similar matrices}

\begin{frame}
\begin{beamercolorbox}[sep=12pt,center]{part title}
\usebeamerfont{section title}
\insertsection\par
\end{beamercolorbox}
\end{frame}

\begin{frame}{Similar matrices}
\begin{definition}
	Two matrices $A$ and $B$ are \emph{similar} iff there exists an invertible $P$ such that
	\begin{equation*}
	A = PBP^{-1}
	\end{equation*}
	and we write $A\sim B$.
\end{definition}\vfill
\begin{example}
	Every diagonalisable matrix is similar to a diagonal matrix.
\end{example}
\end{frame}

\begin{frame}{Trace of a matrix}
\begin{definition}
	The trace of a matrix is the sum of its diagonal elements:
	\begin{equation*}
	tr(A) = \sum_{i=1}^n a_{ii}
	\end{equation*}
\end{definition}
\begin{theorem}
	If $A\sim B$ then $tr(A) = tr(B)$.
\end{theorem}
\end{frame}

\section{Dynamical systems}

\begin{frame}
\begin{beamercolorbox}[sep=12pt,center]{part title}
\usebeamerfont{section title}
\insertsection\par
\end{beamercolorbox}
\end{frame}

\begin{frame}{Dynamical systems}
\begin{definition}
	A \emph{dynamical system} consists of a function $\alpha(t)$ that prescribes how the state of the system changes over time. 
\end{definition}
\begin{definition}
	A \emph{discrete linear dynamical system} consists of a sequence of vectors
	\begin{equation*}
	x_0, x_1, x_2, \dots, x_k, \dots
	\end{equation*}
	such that $x_{k+1} = Ax_{k}$ for some matrix $A$.
\end{definition}
\end{frame}

\begin{frame}{Long term behaviour using eigenvectors}
If $x_0$ is a linear combination of the eigenvectors $v_{\lambda_i}$ of $A$ then 
\begin{equation*}
x_k = A^k x_0 = A^k\left(\sum_{i=1}^n b_i v_{\lambda_i}\right) = \sum_{i=1}^n b_i A^k\left(v_{\lambda_i}\right) = \sum_{i=1}^nb_i(\lambda_i)^kv_{\lambda_i}
\end{equation*}
and so the long-term behaviour is determined by the limits:
\begin{equation*}
\lim_{k\to \infty} (\lambda_i)^k
\end{equation*}
\end{frame}

\begin{frame}{Dominant eigenvalue}
\begin{definition}
	If a is a square matrix then a \emph{dominant eigenvalue $\lambda_{max}$} is one for which $|\lambda_{max}|>|\lambda_i|$ for all other eigenvalues $\lambda_i$.
\end{definition}
\begin{equation*}
x_k = \sum_{i=1}^n b_i (\lambda_i)^k v_{\lambda_i} \approx b_i (\lambda_{max})^k v_{\lambda_{max}}
\end{equation*}
and we can read off the long term behaviour. E.g.
\begin{itemize}
	\item if $|\lambda_{max}|< 1$ then the system converges to $0$
	\item if $|\lambda_{max}| = 1$ then the system converges to $b_iv_{\lambda_{max}}$
	\item if $|\lambda_{max}| = -1$ then the system oscillates between $\pm b_iv_{\lambda_{max}}$
	\item if $|\lambda_{max}>1|$ then the system diverges
\end{itemize}
\end{frame}

\begin{frame}{Examples}
\begin{example}
	Find a formula for $x_k$ if $x_{k+1} = Ax_k$,
	\begin{equation*}
	x_0 = \left[
	\begin{matrix}
	1\\
	-1
	\end{matrix}
	\right] \text{ and } A = \left[
	\begin{matrix}
	2&0\\
	3&-1
	\end{matrix}
	\right]
	\end{equation*}
\end{example}
\begin{example}
	Estimate the long term behaviour of the dynamical system with
	\begin{equation*}
	x_0 = \left[
	\begin{matrix}
	100\\
	40
	\end{matrix}
	\right] \text{ and }
	A = \left[
	\begin{matrix}
	\frac{1}{2}&\frac{1}{4}\\
	2&0
	\end{matrix}
	\right]
	\end{equation*}

\end{example}
\end{frame}

\section{Markov chains}

\begin{frame}
\begin{beamercolorbox}[sep=12pt,center]{part title}
\usebeamerfont{section title}
\insertsection\par
\end{beamercolorbox}
\end{frame}

\begin{frame}{Markov chains}
A \emph{Markov chain} consists of:-\vfill
\begin{itemize}
	\item a finite set of \emph{states} $x_1$, $x_2$, \dots, $x_n$\vfill
	\item a repeated \emph{transition interval} at the end of which the system transitions between states\vfill
	\item a \emph{not necessarily deterministic rule} for predicting the probability that the system will transition into a certain state
	\begin{itemize}
		\item {\bf This probability only depends on the current state.}
		\item (Not the entire history of the chain.)
	\end{itemize}
\end{itemize}
\end{frame}

\begin{frame}{Markov transition matrices}
This means that a Markov chain is described by a \emph{transition matrix} $A$ such that
\begin{align*}
A_{ij} &= \mathbb{P}(X_1 = j | X_0 = i)\\
&= \text{ the probability that the next state will be $j$ }\\
&\hspace{0.5cm} \text{ given that the current state is $i$}
\end{align*}
Therefore:-
\begin{itemize}
	\item all of the entries are between $0$ and $1$
	\begin{itemize}
		\item (I.e. they are probabilities.)
	\end{itemize}
	\item in any column the sum of the entries is $1$
	\begin{itemize}
		\item (The system must be in one of the states at all times.)
	\end{itemize}
\end{itemize}
\end{frame}

\begin{frame}{Steady state}
	\begin{definition}
	A \emph{steady state vector for a matrix $A$} is an eigenvector with eigenvalue $1$.
	\end{definition}\vfill
 	(In other words it is a \emph{non-zero} vector $v$ such that $Av = v$.)\vfill
 	\begin{definition}
 	A \emph{steady state vector for a Markov chain} is a steady state vector for the transition matrix of the Markov chain such that the entries sum to $1$.
 	\end{definition}
\end{frame}

\begin{frame}{Regular Markov chains}
\begin{definition}
A matrix $A$ is \emph{regular} iff there exists a positive integer $k$ such that all of the entries of $A^k$ are \emph{strictly positive}.
\end{definition}
\begin{theorem}
If the transition matrix of a Markov chain is regular then it has a steady state vector.
\end{theorem}
\begin{example}
	The converse is false! E.g. the following matrix has two steady state vectors: 
	\begin{equation*}
	A = \left[
	\begin{matrix}
	1 & 0\\
	0 & 1
	\end{matrix}
	\right]
	\end{equation*}
\end{example}
\end{frame}

\begin{frame}{Example}
\begin{example}
	Calculate and interpret $x_3$ given that
	\begin{equation*}
	x_0 = \left[
	\begin{matrix}
	\frac{1}{2}\\
	\frac{1}{2}
	\end{matrix}
	\right] \text{ and }
	A = \left[
	\begin{matrix}
	0 & \frac{1}{4}\\
	1 & \frac{3}{4}
	\end{matrix}
	\right]
	\end{equation*}
\end{example}
\begin{example}
	What is the probability that we are in state $1$ after two iterations if
	\begin{equation*}
	x_0 = \left[
	\begin{matrix}
	1\\
	0\\
	0
	\end{matrix}
	\right] \text{ and }
	A = \left[
	\begin{matrix}
	\frac{1}{2} & \frac{1}{4} & \frac{1}{4}\\
	0 & \frac{1}{2} & \frac{1}{4}\\
	\frac{1}{2} & \frac{1}{4} & \frac{1}{2}
	\end{matrix}
	\right]
	\end{equation*}
\end{example}
\end{frame}

\begin{frame}{Examples}
\begin{example}
	Find a steady state vector for 
	\begin{equation*}
	\left[
	\begin{matrix}
	0 & \frac{1}{4}\\
	1 & \frac{3}{4}
	\end{matrix}
	\right]
	\end{equation*}
\end{example}
\begin{example}
	Find a steady state vector for 
	\begin{equation*}
	\left[
	\begin{matrix}
	\frac{1}{2} & \frac{1}{4} & \frac{1}{4}\\
	0 & \frac{1}{2} & \frac{1}{4}\\
	\frac{1}{2} & \frac{1}{4} & \frac{1}{2}
	\end{matrix}
	\right]
	\end{equation*}
\end{example}
\end{frame}

\section{PageRank}

\begin{frame}
\begin{beamercolorbox}[sep=12pt,center]{part title}
\usebeamerfont{section title}
\insertsection\par
\end{beamercolorbox}
\end{frame}

\begin{frame}{Simplified Google PageRank}
The core part of Google PageRank (mainly due to Larry Page) is calculated recursively:-\vfill
\begin{itemize}
	\item Suppose we have $n$ pages: $P_1$, $P_2$ \dots $P_n$\vfill
	\item denote the PageRank of $P_i$ by $PR(i)$\vfill
	\item then (recursively) define
	\begin{equation*}
	PR(i) = \sum_{j=1}^n \frac{PR(j)\cdot L_{ji}}{Out(j)}
	\end{equation*}
	where $L_{ji}$ is the number of links from $P_j$ to $P_i$ and $Out(j)$ is the total number of links on page $j$.
\end{itemize}\vfill
At first glance it is not even clear that this is well-defined\dots
\end{frame}

\begin{frame}{Steady state vector}
To work out the PageRank for each site:-
\begin{itemize}
	\item Form the matrix with entries $A_{ij} = \frac{L_{ji}}{Out(j)}$.
	\item Find a steady state vector for this system.
\end{itemize}
\end{frame}

\begin{frame}{PageRank Example}
\begin{example}[Simplified PageRank]
	Suppose that the internet consists of three(!) pages $P_1$, $P_2$ and $P_3$.
	Suppose further that:-
	\begin{itemize}
		\item there are two links from $P_1$ to $P_2$
		\item there is one link from $P_1$ to $P_3$
		\item there is one link from $P_2$ to $P_1$
		\item there is one link from $P_2$ to $P_3$
		\item there is one link from $P_3$ to $P_2$
	\end{itemize}
	Then calculate the (relative) PageRanks for $P_1$, $P_2$ and $P_3$.
\end{example}
\end{frame}

\begin{frame}{Example}
\begin{example}
If possible diagonalise
\[
\left[
\begin{matrix}
3&-4&2\\
1&-2&2\\
1&-5&5
\end{matrix}
\right]\]
\end{example}

\end{frame}

\end{document}